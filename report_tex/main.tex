\documentclass[a4paper, twoside, 10pt]{extreport}
\usepackage{extsizes}

\usepackage{csquotes}
\usepackage{filecontents}
\usepackage[english]{babel}
\usepackage{subcaption}
\usepackage[utf8]{inputenc}
\usepackage[T1]{fontenc}
\usepackage{amssymb}
\usepackage{threeparttable}
\usepackage{setspace}
\usepackage{yfonts}
\usepackage{float}
\usepackage{multicol}
\usepackage{tikz}
\usepackage{parskip}
\usetikzlibrary{positioning,calc,automata,circuits.ee.IEC,chains,fit,shapes}
%% Sets page size and margins
\usepackage[a4paper,top=3cm,bottom=2cm,left=3cm,right=3cm,marginparwidth=1.75cm]{geometry}
\usepackage[backend=biber,style=ieee,uniquename=init,giveninits=true]{biblatex}
%% Useful packages
\usepackage{amsmath}
\usepackage{graphicx}
\usepackage{listings}
\usepackage[colorinlistoftodos]{todonotes}
\usepackage[colorlinks=true, allcolors=blue, breaklinks]{hyperref}
\usepackage{nameref}
\usepackage{minted}
\usepackage{cprotect}
\usepackage{amsmath}
\usepackage{algorithm}
\usepackage{algpseudocode}
\urlstyle{same}

\pagenumbering{roman}

%\makeatletter
%\renewcommand*{\thelisting}{\thesection.\arabic{listing}}
%\@addtoreset{listing}{section}
%\makeatother
\counterwithin{listing}{chapter}

\makeatletter
\def\BState{\State\hskip-\ALG@thistlm}
\makeatother
\lstset{
basicstyle=\small\ttfamily,
columns=flexible,
breaklines=true
}

\makeatletter
\let\my@chapter\@chapter
\renewcommand*{\@chapter}{%
  \addtocontents{lol}{\protect\addvspace{10pt}}%
  \my@chapter}
\makeatother

\tikzset{
  op/.style={circle,draw,minimum width=2.5em,inner sep=1pt,font=\small},
  buf/.style={rectangle,draw,minimum height=2.5em,minimum width=2.5em,font=\small},
  add/.style={op,node contents={$+$}},
  sub/.style={op,node contents={$-$}},
  div/.style={op,node contents={$\div$}},
  div_4/.style={op,node contents={$\div$ $4$}},
  >=stealth
}
\newcommand\drawdh[4][]% diagonal horizontal
  {\draw[#1]
     let \p1=(#2),\p2=(#3) in
     (#2) -- node[above,font=\scriptsize]{#4}
     ({\x1+abs(\y2-\y1)*sign(\x2-\x1)},\y2) -- (#3);%
  }
\newcommand\drawdv[4][]% diagonal vertical
  {\draw[#1]
     let \p1=(#2),\p2=(#3) in
     (#2) -- node[above,font=\scriptsize]{#4}
     (\x2,{\y1+abs(\x2-\x1)*sign(\y2-\y1)}) -- (#3);%
  }

\bibliography{bibs/sample}

\title{Flow DSL: A Domain-Specific Language for Static Streaming Hardware}
\author{Timothy Moores}
% Update supervisor and other title stuff in title/title.tex

\begin{document}
\input{title/title.tex}


\raggedbottom

\begin{abstract}
  \thispagestyle{plain}
Hardware stream processing is often achieved using low-level register transfer languages. Despite these languages introducing complexity to the design process, when designing systems at the register transfer level, the timing of modules must be manually evaluated each time the module is instantiated to ensure all branches of the stream are synchronised. This is due to most commonly available alternatives being too high level for the design of stream processing hardware.
\\ \\
While toolkits for designing stream processing hardware do exist, they are largely inaccessible to individuals and implemented in ways which are not immediately compatible with the concept of stream processing. In this paper we present a domain-specific language for writing such systems in the form of dataflow graphs and a reference implementation of a compiler to transform it to Verilog for synthesis or simulation.
\\ \\
This paper introduces a domain-specific language (DSL) that simplifies hardware stream processing design by representing systems as dataflow graphs. The DSL abstracts away the intricacies of low-level languages while maintaining the necessary precision for hardware design. Additionally, a compiler is provided that transforms the DSL into Verilog, ensuring compatibility with established tools.
\\ \\
Flow DSL was found to be a useful tool for high level synthesis, successfully bridging the gap between abstract dataflow graphs and complex hardware constructs. When compared to hand-written Verilog, the optimisations of Flow DSL can make noticeable resource usage improvements and is significantly shorter to write and modify, and the inherent flexibility and simplicity makes Flow DSL a productive alternative added to the hardware development landscape.


\end{abstract}
\newpage
\renewcommand{\abstractname}{Acknowledgements}
\begin{abstract}
  \thispagestyle{plain}
  \setcounter{page}{2}
  \begin{itemize}
    \item \textbf{Professor Paul Kelly}: For his unwavering support and mentorship throughout the course of this project. His dedication, insights, and countless hours of supervision have been a constant source of guidance and inspiration.
    \item \textbf{Dr.\ Tony Field}: Whose astute advice significantly influenced the trajectory of this project.
    \item \textbf{George Bisbas}: For his generous assistance in familiarizing me with xDSL. His expertise and willingness to share knowledge have been invaluable in navigating the technical aspects of the project.
  \end{itemize}
  Additionally, I would like to express my appreciation to:
  \begin{itemize}
    \item \textbf{Razvan Rusu}, \textbf{Neel Dugar} and \textbf{Sherwin Da Cruz}: Our numerous foosball breaks were not just games; they were essential mental health breaks that kept me grounded and balanced and had a significant role in maintaining my well-being throughout the project.
  \end{itemize}
\end{abstract}

\tableofcontents
\listoffigures
\listoftables
\listoflistings
\newpage

\pagenumbering{arabic}
\clearpage
% Introduction
\chapter{Introduction}
This project centres around the design, implementation and evaluation of a DSL, Flow DSL, for converting static streaming dataflow graphs, such as the one in figure \ref{intro.dataflow}, into hardware. In parallel, the project investigates the potential applications of the static streaming dataflow model in other programming domains and for other targets. The DSL itself is designed to be reconfigurable, easy to learn and intuitive to people who know Python. It is dissimilar to most existing hardware development languages, abstracting away the concept of registers, a clock and other features which may serve to unnecessarily complicate the DSL, as exposing clock logic is not necessary when inputs and outputs arrive and leave every clock cycle. A functionally equivalent Verilog code listing for the can be seen in listing \ref{verilog.roberts_cross}.

\begin{figure}[H]
  \centering
  \begin{subfigure}[b]{0.5\textwidth}\centering
    \caption*{}
    \begin{minted}[numbers=left]{python}
screen_height = 480

r_in = gr.istream(8)
g_in = gr.istream(8)
b_in = gr.istream(8)

def roberts_cross(in0: Node) -> Node:
  last = in0.offset(-1)
  prev_row = in0.offset(-screen_height)
  prev_row_last = last.offset(-screen_height)
  gx = in0 - prev_row_last
  gy = prev_row - last
  return abs(gx) + abs(gy)

r_out = gr.ostream(roberts_cross(r_in), 8)
g_out = gr.ostream(roberts_cross(g_in), 8)
b_out = gr.ostream(roberts_cross(b_in), 8)
    \end{minted}
    \caption{Flow DSL}
  \end{subfigure}%
  \begin{subfigure}[b]{0.5\textwidth}\centering
    \caption*{}
    \begin{tikzpicture}[node distance=1cm,circuit ee IEC]
      % below n
      \node (n) {$[rgb_{-n}, ..., rgb_{-1}, rgb_0, rgb_1, ..., rgb_{n}]$};
      \node (buf_0) [buf,below left=of n,xshift=1.5cm]{$last_{480}$};
      \node (buf_1) [buf,right=of buf_0,xshift=-0.5cm]{$last_1$};
      \node (sub_0) [op, below=of buf_1,xshift=-0.5cm]{$-$};
      \node (abs_0) [op, below=of sub_0,yshift=1cm,xshift=1cm]{abs};
      \node (buf_2) [buf,below right=of n,xshift=-2cm]{$last_{481}$};
      \node (sub_1) [op, right=of sub_0,xshift=2cm]{$-$};
      \node (abs_1) [op, below=of sub_1,yshift=1cm,xshift=-1cm]{abs};
      \node (add_0) [op, right=of abs_0,yshift=-1cm,xshift=-1cm]{$+$};
      \node (y) [below=of add_0] {$[out_{-n}, ..., out_{-1}, out_0, out_1, ..., out_{n}]$};
      \draw[->](n)--(buf_0);
      \draw[->](n)--(buf_1);
      \draw[->](buf_0)--(sub_0);
      \draw[->](buf_1)--(sub_0);
      \draw[->](sub_0)--(abs_0);
      \draw[->](abs_0)--(add_0);
      \draw[->](n)--(buf_2);
      \draw[->](buf_2)--(sub_1);
      \draw[->](n)--(sub_1);
      \draw[->](sub_1)--(abs_1);
      \draw[->](abs_1)--(add_0);
      \draw[->](add_0)--(y);
    \end{tikzpicture}
    \caption{Dataflow Graph}
  \end{subfigure}
  \caption{Roberts Cross edge detector}\label{intro.dataflow}
\end{figure}

\section{Motivation}
Stream processing, the practice of sequentially processing data records at a constant rate as defined by external factors, is increasingly used in diverse fields such as data analytics, digital signal processing, and network security. With the growing amount of data generated by modern applications, efficient and reliable stream processing has become a critical requirement, often necessitating the use of hardware acceleration to meet performance and latency goals \hyphen{} indeed stream processing is one of the most popular applications for hardware acceleration.

Despite the popularity of stream processing, existing hardware design tools suitable for its implementation often suffer from high entry barriers, outdated practices, or convoluted or `boilerplate' filled syntaxes. This creates a challenging landscape for system designers, often pushing them to use low-level Register Transfer Level (RTL) languages like Verilog or VHDL, or to use more general High-Level Synthesis (HLS) tools which often require sacrificing control over timing or the level of optimisation which can be applied. In both cases, the potential loss of fine-grained control crucial to achieving a specific throughput and the prioritization of latency in HLS tools can be problematic for stream processing.

Existing toolkits for stream processing hardware acceleration are often poorly documented and commercially restricted, limiting their accessibility, especially for individual developers and researchers. This project intends to fill this gap with an accessible, user-friendly tool.

Interestingly, the concepts underpinning stream processing show considerable overlap with declarative programming. The notion of reaching definitions which may even be forward references found in declarative languages has many parallels with the static streaming paradigm. The immutability of hardware variables within a hardware clock cycle also aligns with the principles of functional and declarative programming. This overlap suggests the potential for a domain-specific language (DSL) catering to stream processing hardware that borrows principles from declarative programming.


\section{Technical Challenges}
\subsection{General Challenges}
\subsubsection*{Synthesisability}
Designing a new DSL for hardware acceleration comes with the challenge of ensuring synthesisability. The language constructs and the operations expressed must be synthesisable into hardware constructs. This involves ensuring that the language elements can be mapped onto hardware primitives such as gates, flip-flops, memories, and arithmetic units, or to a subset of an RTL-supporting HDL which allows for relatively low level specification of hardware.
\subsubsection*{Retiming Algorithms}
Developing retiming algorithms to optimise the performance of hardware designs is another challenge. Even in the simpler dataflow domain, developing such algorithms is nontrivial, as retiming certain nodes may have knock-on effects on the others. There is also the consideration of balancing latency against resource usage.
\subsubsection*{Evaluation}
A new DSL and the associated IR dialects need extensive testing and evaluation. This requires developing benchmarks and performance metrics, implementing test cases, and possibly also creating hardware prototypes. These activities are time-consuming and require tailoring both to the domain from which the test case comes and the hardware architectures it will be deployed to.

\subsection{Implementation Specific Challenges}
\subsubsection*{Limitations of Python}
While Python will be a powerful tool for the programmer of Flow DSL programs, it presents a number of challenges for writing the DSL, the most major of which being the lack of a strong type system. As polymorphism and the type system in general in Python isn't powerful compared to those in many other languages, a lot more compile time type verification is required when compared to that in other languages, as there is no Python compiler to do the work for us.

\section{Contributions}
\subsection*{Design and implementation of the Flow DSL}
Designing new domain-specific language that simplifies the design of stream processing hardware. The language provides high-level abstractions and an intuitive syntax, making it easy for designers to express complex algorithms and designs as a dataflow graph.

\subsection*{Analysis and implementation of retiming algorithms}
Designing retiming algorithms that cater specifically to the needs of dataflow-based stream processing. These algorithms are distinguished in their ability to allow for full cycles, either backwards or forwards. This feature stands in contrast to many existing languages that only support accumulators or referencing previous values without the possibility of forward references or getting future values. By supporting a broader range of retiming operations, our algorithms offer increased flexibility and optimization potential in hardware design, enhancing the efficacy of the overall system.

\subsection*{Development and implementation of xDSL dialects}
Developing and implementing dialects within the xDSL platform, which may be reusable for other HLS tools, which facilitate a multi-stage lowering structure. Currently, we support one high-level, two mid-level, and one low-level dialect, the latter of which can be translated into Verilog, a common hardware description language. Although only Verilog is currently supported as the target language, the advantage of the lowering design is its flexibility. The independent IR allows for easy retargeting to other hardware description languages or simulators in the future. This provides a path for extending the applicability of our DSL to a broader range of hardware design workflows without needing extensive redesign.

\subsection*{Reduction of `boilerplate' in hardware synthesis}
One of the key contributions of this project is the significant reduction of boilerplate code that is typically required when designing hardware using traditional RTL languages or HLS tools. With our DSL, users can focus on expressing their algorithms as a dataflow graph, while the tool takes care of the lower-level hardware implementation details. As the compiler is written in Python, there is little setup needed on a machine running it, and little extra code needed to get started. This results in more efficient hardware design practices, faster prototyping, and ultimately shorter time-to-market.

\subsection*{Creation of a framework with high optimisation potential}
An issue with many existing HLS frameworks is that there is limited scope for optimisation, as bit widths and timing of all variables is generally set at compile time. The Flow DSL is not subject to these limitations, as the timing is only fixed on inputs and the bit widths are only fixed on inputs and outputs at the beginning of the compilation process. This allows for more flexibility and optimisation during the compilation process, resulting in more efficient hardware designs.

\subsection*{Evaluation and testing of the DSL}
The DSL was evaluated and compared to existing popular HDL languages which allow full control of the clock to have a throughput of one value per input or output per clock cycle. As performance differs greatly between applications due the the multipliers, logic elements and buffers available on a given FPGA, a more widely applicable method of objective comparison had to be selected. Evaluation was also done on the more subjective features of the language, such as its verbosity, ease of writing and ease of modification.
% Setting
\chapter{Background}
This section aims to introduce the reader to concepts relevant to the Flow DSL. This is split into four sections: High Level Synthesis (\ref{hls}), the Dataflow Programming Model (\ref{dataflow}), Embedded Domain-Specific Languages (\ref{edsl}) and Intermediate Representation Frameworks (\ref{irf}).

\section{Hardware Design Languages}\label{hls}
Hardware Design Languages (HDLs) are specialised programming languages used to design digital logic systems, such as field-programmable gate arrays (FPGAs) and application specific integrated circuits (ASICs). HDLs allow designers to describe the structure, functionality, and timing of their designs in a high-level and abstract manner, which can then be synthesised into physical designs by tools such as logic synthesisers and place-and-route tools.

\subsection{Register Transfer Level}
Designing hardware at a register transfer level (RTL) allows designers to express at a relatively low level the operations which are to happen in hardware synthesised from it. This involves using registers, which are a flip-flop and can be updated at each clock cycle, or wires, which are assumed to update instantly: set-up times are not considered at this point. This abstraction allows RTLs to describe the movement and modification of digital signals as they are `transferred' between registers. An RTL description is usually converted into a gate-level description by a synthesis tool, which can then be used for further processing like placement, routing, and bitstream generation, and many HLS compilers will target an RTL rather than synthesising the design themselves. RTL design is generally performed in languages such as Verilog and VHDL, which are considered lower level than many other types of HDL.

\subsection{High Level Synthesis}
High-Level Synthesis (HLS) is a method of designing hardware systems using a higher level of abstraction than RTL. HLS tools take descriptions of hardware described in high-level programming languages like C or C++, or in DSLs which allow hardware to be described in many different ways via many different abstractions, and synthesise them into RTL designs. This allows hardware designers to express complex algorithms succinctly and in a more software-like manner, reducing the time it takes to design and verify a system.

HLS tools generally cannot be used to create any given circuit, and therefore the languages are not technically HDLs. This is because it is a circuit which has the behaviour described in the program which is generated by the compiler, rather than an exact 1:1 mapping of variables and operations to wires and registers. To simplify the compilation process and limit the language to only concepts which are transferable to hardware, subsets or variants of these languages such as Single-Assignment C \cite{sa-c}, or the subset used by ROCCC \cite{5474060} are used.  Some languages used for HLS abstract away the concept of a clock, with the timing decided by the compiler; others keep the clock and leave it to the user to specify the timing of their design, which may or may not be tweaked by the compiler.

\subsubsection{Formal Verification}\label{formalverif}
Formal verification is an automated method for rigorously checking the correctness of a program. When applied to HLS, this program is a piece of hardware. It involves the use of mathematical methods to ensure that a given design adheres to its specified behaviour, eliminating the need for exhaustive simulation tests.

Formal verification can ensure that the functionality of a design matches its specifications, thereby reducing the likelihood of undetected design errors causing issues later on. Additionally, formal verification can also significantly speed up the testing process of a design, as simulations often require significant time to run. Simulations may not even cover all possible edge cases, whereas formal verification will.

Formal verification is easier to perform on programs in specific forms, and some HLS tools can perform formal verification on both the input and output programs. As Herklotz, Pollard, Ramanathan, and Wickerson discovered in their 2021 paper `Paper' \cite{formal_verif}, it is not uncommon for even commercially supported HLS tools such as Vivado \cite{vivado} to apply optimisations incorrectly, leading to the functionality of the synthesised hardware design not matching the HSL specification. By using a language and intermediate representation where formal verification are more straightforward it is possible to prove that an optimised design matches its high level specification.

Formal verification is especially useful in hardware as there are fewer error recovery and debugging mechanisms when compared to software \hyphen{} when in software one may just deploy a fix for reported malfunctioning software, it is unlikely to be immediately apparent that hardware is malfunctioning at all, and even then finding what the issue is is likely to be more of a challenge due to the lack of debugging software. On fabricated hardware, fixing bugs will result in the whole fabrication system, including any dies and tooling, needing to be replaced.

\subsection{Related Work in Hardware Development Languages}
\subsubsection{SystemVerilog}
SystemVerilog \cite{8299595} is a hardware description and verification language that extends and improves upon the Verilog language similarly to how C++ extends C in the software world. It offers a number of advanced features for modelling hardware, such as an enhanced type system when compared to Verilog, interfaces, and design hierarchy. SystemVerilog also introduces a comprehensive set of verification constructs, allowing for complex testbenches and verification environments to be built alongside the design itself. This makes it a popular choice for complex, safety-critical designs which require design at a register transfer level and necessitate formal verification.

\subsubsection{Survey of Domain-Specific Languages for FPGA Computing}
In the paper `Survey of Domain-Specific Languages for FPGA Computing' \cite{7577380}, Kapre and Bayliss present a comprehensive overview of the current state of DSLs targeted towards FPGA computing. They note that hardware languages lag far behind software languages in terms of the level of abstraction they provide as well as the tooling around these languages which enables developers be more productive when writing their code.

They explore the advantages and disadvantages of using DSLs over traditional HDLs, and identify trends and areas of ongoing research in the field. One key finding is the rise of spatial programming, a paradigm that treats hardware as a two-dimensional space and allows designers to express parallelism and locality explicitly. This approach is a departure from traditional sequential programming and offers significant performance advantages for hardware designs.

The survey also highlights the increasing sophistication of compiler and synthesis technologies, which are becoming more adept at translating high-level constructs into efficient hardware implementations. These advances are enabling DSLs to deliver competitive performance while offering higher productivity than traditional HDLs.

\subsubsection{LLHD}
LLHD (Low Level Hardware Description) \cite{mlir} is a language designed for digital hardware modelling. It provides a middle ground between high-level HDLs like SystemVerilog and gate-level netlists. Its lower level of abstraction enables detailed control of signal timing and operations, but it is more amenable to compiler analysis and transformations than traditional HDLs. This makes it particularly suited to verification, simulation, and as a target for HLS tools.

\subsubsection{CIRCT}
The Circuit IR Compilers and Tools project (CIRCT) \cite{circt} aims to apply MLIR and LLVM techniques to the domain of digital design tools. Its goal is to develop open-source tools that support a variety of hardware design and verification tasks. These tools work with a variety of input languages and aim to provide efficient and high-quality output for multiple backends. One of the key aspects of CIRCT is its support for and development of the LLHD intermediate representation, which it uses to enable sophisticated transformations and optimizations of digital designs.

\section{Dataflow Programming Model}\label{dataflow}
The dataflow programming model is a declarative programming model wherein the program is described as a directed graph. This structures the program as operations, which are represented as the vertices of the graph, and dependencies, which are represented as the directed edges of the graph. By specifying the flow of data through the program, rather than specifying an order for computations to be done in, as in imperative programming, each node of the graph can be executed concurrently, leading to performance improvements in parallel and distributed computing environments where hardware allows for this.

Within this graph, the edges represent the data itself, whereas the vertices represent an operation upon the vertices which are connected to it, outputting a value on the edges connected from it. Not all vertices have operands or results, however, as nodes can be external inputs or outputs, as well as constant values. Each node and branch of the graph can be computed in isolation from other nodes or branches, therefore there is a lot of potential for parallelism.

\subsection{In Software}
In the context of software, the dataflow programming model is generally used on a high level and often in applications which use huge amounts of data, such as cloud data movement \& processing and machine learning. While these applications, especially machine learning, are occasionally compiled to hardware, they are far more often run on off-the-shelf processing units. These may be optimised for parallelism, such as GPUs, or for general use, such as CPUs, however in either case the compiled program is not guaranteed to retain the same parallelisability as the original graph due to the limitations of the hardware they are running on. Optimising compilers for imperative code often use a dataflow programming model internally, as it can be easier to perform optimisations by removing redundant computations, reordering instructions, eliminating dead variables and more when the program is structured as a graph of its operations and their dependencies, with no implicit ordering between independent computations.

\subsection{In Hardware}
In the context of hardware, the dataflow programming model is much more similar to RTL code than it is to machine code for software. This is because there are far fewer restrictions on the number of parallel computations which can occur in hardware, being limited only on FPGAs by the number of logic elements available. Without the concept of a `current instruction' which changes every clock cycle, which is controlled by a program counter in a usual CPU, each node of the dataflow graph can be computed as and when all of its dependencies are available, or even on every clock cycle.

\subsection{Static Streaming Dataflow}
Static streaming dataflow is a specific variant of the dataflow programming model that focuses on streaming computations through a dataflow graph at a high throughput. It is commonly used in the design of signal processing systems and other applications where data is continuously processed in real-time. In a static streaming dataflow graph, it is assumed that inputs will be received at a regular interval, equivalent to the `enable' signal being held high in a non-streaming dataflow hardware application. As there is an assumption of a new input every cycle, any operation that may introduce a stall, such as reading or writing to a synchronous RAM block, must be used with caution to ensure it won't produce a perpetual backlog. While in some applications it may be acceptable to skip inputs in the case of a stall, this project will focus on the applications where every input must generate an output.

\subsection{Related Work}
\subsubsection{Single Assignment C and the Cameron Project}
SB. Scholz' Single Assignment C \cite{sa-c} has been used in by Rinker et al. \cite{920828} in a subset of the Cameron project \cite{najjar1998cameron}, for compiling acyclic dataflow graphs to hardware, highlighting the single assignment model's association with both the dataflow programming model and high level synthesis. While Single Assignment C is not immediately compatible with the cyclic dataflow programming model, the combination of the Cameron project's dataflow graph compiler and the concepts brought forward by Single Assignment C provide a good introduction to how the dataflow programming model can be seen as adjacent to hardware.

\subsubsection{Lustre}
Lustre \cite{lustre} is a declarative, synchronous, streaming dataflow language for programming real-time systems in software, specifically designed for systems that interact with their environment at a pace determined by the environment, for instance in the critical parts of Airbus aircrafts. As it is a streaming dataflow language, the length of the input vector is assumed to be infinite \hyphen{} indeed in the applications of critical system monitoring it should be running the entire time the system is powered up.

Lustre allows for inputs of both integer and vector form, and has a number of helper functions which are often useful in real time systems programming using real world data, such as the dot product operator \lstinline|.*|, and additional boolean operations such as \lstinline|implies|, where \lstinline|implies(a, b)| returns true if \lstinline|b| is true or \lstinline|a| is false. This is useful in systems programming to assert `whenever \lstinline|a| is high, \lstinline|b| should also be high'. It also allows the previous value of a variable to be retrieved with \lstinline|pre|, shown together with the dot product operator in listing \ref{lustre.mvg_avg}.

Lustre, along with the dataflow programming model in general, is especially well suited to formal verification as shown by Hagen \& Tinelli in their 2008 paper `Scaling up the formal verification of Lustre programs with SMT-based techniques' \cite{lustreverification}. Formal verification is especially useful for hardware, as explained in section \cite{formal_verif}.

\renewcommand\theFancyVerbLine{\arabic{FancyVerbLine}}
\begin{listing}[H]
  \begin{minted}[numbers=left]{text}
node MovingAverage(t_0: int[5], t_1: int[5]) returns (output: real);
var
  input: int;
  total: real;
  div_t: real;
const
  avg_window: int = 4;
let
  input = t_0 .* t_1;
  t_0 = pre(input, avg_window) - input;
  total = pre(total, 0.0) + t_0
  div_t = total / real(avg_window);
  output = div_t;
tel
  \end{minted}
  \caption{A Lustre implementation of a dot product moving average}\label{lustre.mvg_avg}
\end{listing}

\subsubsection{Maxeler MaxJ}
MaxJ \cite{maxj} is a high-level, static streaming dataflow language developed by Maxeler Technologies for implementing dataflow computing for hardware acceleration. MaxJ's MaxCompiler uses a Java-like syntax and runs in the Java runtime environment, allowing users to use the features and conveniences of Java while expressing their applications as a highly parallelised and pipelined streaming dataflow graph. The key feature of MaxJ is its ability to express dataflow computations in a way that can be compiled, via effective and powerful optimisations, to hardware designs that are capable of achieving high-performance. This is particularly important in fields such as finance, oil and gas exploration, scientific research, and data analytics where FPGAs can offer significant speedup over traditional CPU and GPU architectures.

MaxJ is closed source and not easy to get information on, however it is designed to work for compilation to and high-performance \& predictable execution on Maxeler's hardware platforms Maxeler's own FPGA products, as the data movement and processing can be optimised in advance based on the static analysis of the code and the architecture of Maxeler's FPGAs.

MaxJ uses what it calls `Kernels' to perform operations on the stream, and the previous value in the stream can be accessed using `Registers'. There are also accumulators and reduction methods built into the language, which allows for simple cycles to be constructed, and the "offset" function, which allows past values of already declared `DFEVar's, but not forward references. By nature of being embedded in Java, which requires a large amount of `BoilerPlate' code, MaxJ can be fairly unwieldy even for simple designs, such as the moving average kernel shown in listing \ref{maxj.mvg_avg}.

\renewcommand\theFancyVerbLine{\arabic{FancyVerbLine}}
\begin{listing}[H]
  \begin{minted}[numbers=left]{java}
public class MovingAverageKernel extends Kernel {
    public MovingAverageKernel(KernelParameters parameters) {
        super(parameters);

        int avgWindow = 4;
        Stream<Input> in0 = io.input("in0", dfeUInt(8));
        Register<DFEVar> t_0 = control.count.simpleCounter(1);
        DFEVar c1 = constant.var(avgWindow);
        Loop iterate = control.count.makeCounter(avgWindow);
        DFEVar sum = dfeUInt(8);
        sum <== stream.offset(in0, iterate) + sum;
        t_0 <== sum - in0;
        Accumulator acc = Reductions.accumulator;
        Params paramsInt = acc.makeAccumulatorConfig(dfeInt(10));
        DFEVar total = acc.makeAccumulator(a, paramsInt);
        DFEVar div_t = total / c1;
        io.output("out0", div_t, dfeInt(8));
    }
}
\end{minted}
  \caption{A Maxeler MaxJ implementation of a moving average}\label{maxj.mvg_avg}
\end{listing}


\subsubsection{Design and Optimisation of Behavioral Dataflows}
Liu's 2019 dissertation, `Design and Optimisation of Behavioral Dataflows' \cite{liu2019design}, discusses design and optimisations of dataflow hardware systems. While the first few chapters on pin multiplexing, DSE and inter-module connections are beyond the scope of this project, the chapters on pure dataflow optimisation, ie from 6 onwards, are highly relevant. These chapters investigate and propose algorithms for how different branches of a dataflow graph should be optimised and retimed differently to result in an optimal final dataflow graph.

\subsubsection{A Second Opinion on Data Flow Machines and Languages}
`A Second Opinion on Data Flow Machines and Languages' \cite{secondopinion} provides a thorough analysis of dataflow computing as of the time time paper was published \hyphen{} 1982. Due to the factual nature of dataflow programming, most of the observations in the paper are still relevant to this day. One of the primary observations of the paper is that while dataflow models are intuitively attractive and conceptually simple, their implementation can be complex and difficult. The authors argue that the key to practical dataflow machines lies in the careful management of the `tokens' that represent data in the system. These `tokens' are considered in a time-insensitive environment as is often present in software, there is no restrictions on buffering, therefore a `token' may be held at the input to a node indefinitely while the other inputs to that node become ready. Furthermore, the authors propose improvements to dataflow languages that aim to better expose opportunities for parallelism while also providing tools for managing the complexities of synchronization and communication.

\section{Embedded Domain-Specific Languages}\label{edsl}
A domain-specific language (DSL) is a programming language specialised to a particular domain \cite{CACCIAGRANO2020100020}. DSLs are often used to improve the productivity of programmers by providing a more concise and expressive way to write code. Unlike external DSLs, which have their own syntax and type system, embedded DSLs `piggyback' off a host language, with the compiler or interpreter for the DSL usually being available as a package or library for that language. The host language is most frequently a general purpose language, where general purpose languages differ from domain-specific languages in that they may be used in many different domains.

\subsection{Related Work}
\subsubsection{Chisel}
Chisel is an embedded DSL for writing hardware designs in Scala developed at University of California, Berkeley. Despite being an embedded DSL, the exceptional support Scala provides for creating embedded DSLs allows for Chisel to be an extremely expressive Hardware Development Language. It provides finer control over the hardware design and offers a more direct representation of the underlying circuitry than HLS tools, however, Chisel's integration with Scala enables a higher level of abstraction and productivity through the use of modern programming language features than afforded by most HDLs such as SystemVerilog or VHDL. Chisel does not use Scala's core types, instead using Chisel types derived from `Bits'. Using its own assignment operator along with the powerful reflection tools within the Scala compiler and Java Virtual Machine, the DSL backend captures the state of each wire and register. This state is then utilised by the Chisel backend to generate the corresponding low-level hardware description. The backend can output the design in Verilog and FIRRTL amongst others, which can be further processed by EDA tools for synthesis, simulation, and implementation. There is also a simulation runtime called `Treadle' and a testing \& formal verification framework called `ChiselTest' built into the library, which allows for testing directly from Scala and therefore even greater productivity. An example of parametrisation in Chisel allowing for an inner product of vectors of parametrised width can be found in listing \ref{chisel.innerproduct}.

Chisel differs from Flow DSL in that Chisel provides a more direct representation of the underlying hardware than Flow DSL, which abstracts a significant proportion of hardware concepts away, however Chisel does have similar variable capture mechanisms to those used by Flow DSL and explained in section \ref{varname}, as well as syntax generally similar to the host language.

\renewcommand\theFancyVerbLine{\arabic{FancyVerbLine}}
\begin{listing}[H]
  \begin{minted}[numbers=left]{scala}
class InnerProduct(width: Int) extends Module {
  val io = IO(new Bundle {
    val vectorA = Input(Vec(width, UInt(32.W)))
    val vectorB = Input(Vec(width, UInt(32.W)))
    val result = Output(UInt(32.W))
  })
  
  io.result :=
    io.vectorA.zip(io.vectorB)
              .map(((_:UInt)*(_:UInt)).tupled)
              .reduce(_+_)
}
\end{minted}
  \caption{A Chisel implementation of an inner product}\label{chisel.innerproduct}
\end{listing}

\subsubsection{Spatial}
Spatial is an embedded DSL developed for HLS hardware accelerator applications, using Scala as a host language. It provides a higher level of abstraction compared to traditional HLS tools while offering finer control over the hardware design than general purpose HLS languages. Spatial allows for the description of computations in a high level using a \lstinline|accel| block using a Scala anonymous function, decoupling the design from specific architectural details. Spatial provides constructs for parallel, sequential and pipelined architectures from a single \lstinline|accel| block, simply by tweaking the config. The \lstinline|Range.par(Int)| function in listing \ref{spatial.innerproduct}, taken from Spatial's tutorial \cite{spatialtut} takes the \lstinline|reduce| calculation and parallelises it, in the case of the example by a factor of 2 and 4 in each usage. The example also shows SRAMs being declared within a loop, which allows for multiple banks of SRAMs to be declared to match the parallelism factor.

Spatial is similar to Flow DSL in that it allows for fully pipelined accelerated sections of a design, however while Flow DSL is designed for end-to-end design in dataflow graphs, Spatial is designed to only have parts of a design be pipelined \hyphen{} others may be parallelised, sequential or partially pipelined.

\renewcommand\theFancyVerbLine{\arabic{FancyVerbLine}}
\begin{listing}[H]
  \begin{minted}[numbers=left]{scala}
Accel {
  // *** Parallelize tile
  x := Reduce(Reg[T](0))(len by tileSize par 2){tile =>
    // *** Declare SRAMs inside loop
    val s1 = SRAM[T](tileSize)
    val s2 = SRAM[T](tileSize)
    
    s1.load(d1(tile::tile+tileSize))
    s2.load(d2(tile::tile+tileSize))
    
    // *** Parallelize i
    Reduce(Reg[T])(0)(tileSize by 1 par 4){i => 
      s1(i) * s2(i)
    }{_+_}
  }{_+_}
}
\end{minted}
  \caption{Spatial's provided implementation of an inner product}\label{spatial.innerproduct}
\end{listing}

\subsubsection{Keras}
Keras is a high level API for describing models for deep learning frameworks; most often for Tensorflow. Keras allows for machine learning models to be programmatically declared via a DSL embedded in Python, including through modular design. While Keras does not support cyclic graphs in most cases, it can be extended with recurrent layers such as LSTM (Long Short-Term Memory) or GRU (Gated Recurrent Unit), which have internal cycles which are hidden from the user, in a similar way to the Accumulator node in a number of dataflow languages. This differs from Flow DSL in that Flow DSL can have either hidden \& implicit or explicit \& external cycles, whereas Keras only allows for hidden \& implicit cycles. Keras also allows for networks to be described functionally, as in listing \ref{keras.dataflow.sequential} or imperatively, as in listing \ref{keras.dataflow.programatic}. These construction methods essentially produce the same graph given the same layers.

\renewcommand\theFancyVerbLine{\arabic{FancyVerbLine}}
\begin{listing}[H]
  \begin{minted}[numbers=left]{python3}
model = tf.keras.Sequential()
model.add(tf.keras.layers.Dense(64, activation='relu', input_shape=[256, 256, 3]))
return model.compile(optimizer='adam', loss='categorical_crossentropy')
  \end{minted}
  \caption{A Keras model constructed functionally}\label{keras.dataflow.sequential}
\end{listing}

\renewcommand\theFancyVerbLine{\arabic{FancyVerbLine}}
\begin{listing}[H]
  \begin{minted}[numbers=left]{python3}
inputs = tf.keras.layers.Input(shape=[256, 256, 3])
outputs = keras.layers.Dense(64, activation='relu')(inputs)
return tf.keras.Model(inputs=inputs, outputs=outputs)
  \end{minted}
  \caption{A Keras model constructed programmatically}\label{keras.dataflow.programatic}
\end{listing}

\section{Intermediate Representation Frameworks}\label{irf}
Intermediate Representation (IR) frameworks play a critical role in the design and implementation of modern compilers. They offer a high-level abstraction of code that is between the source language the target language, often a high level software language and an assembly language, without being too strongly tailored to either. This layer of abstraction simplifies various phases of compilation, including analysis, optimization, and code generation, and facilitates retargetability and code portability.

The IR framework used during compilation can have a significant impact on the compiler's effectiveness and the quality of the generated code, due to its support for optimisation passes \hyphen{} both defined by the complier's author and by others who have contributed to the framework's codebase. By affecting the compiler's ability to optimise code, the IR framework plays a significant role in determining the final program's performance. An ideal IR should be compact, easy to generate and transform, capable of representing all source language constructs, and effective for target machine code generation.

IRs are typically designed to represent the flow of data in a program, and different IRs, often called `dialects', are used for different phases of compilation, including high-level, mid-level, and low-level IRs. High-level IRs are closer to the source language, low-level IRs are closer to the machine code, and mid-level IRs bridge the gap between them, however there may be many of each of high-level IRs, low-level IRs and mid-level IRs.

\subsection{Related Work}
\subsubsection{MLIR}
Multi-Level Intermediate Representation (MLIR) \cite{mlir} is an IR framework developed by the LLVM project. MLIR is designed to bridge the gap between high-level, domain-specific representations and low-level, target-specific representations. The framework provides a common infrastructure to build multiple levels of abstraction into a single unified IR, addressing the growing need for multiple abstraction levels in modern software stacks.

One of the key features of MLIR is its support for a mix of high-level and low-level dialects in the same module, thus providing flexibility in optimizing different parts of a program at different abstraction levels. This simplifies the design and implementation of complex compiler transformations that operate across abstraction boundaries.

MLIR is based on a flexible type system and operation-centric design, which makes it simple to define new types and operations. It provides a pattern-based rewriting infrastructure that reduces the difficulty of implementing transformations and analyses.

MLIR promotes a progressive lowering approach where high-level constructs are gradually lowered to simpler constructs, with optimizations performed at each level. This enables high-level optimizations that are aware of the semantic context and low-level optimizations that are close to the target language.

\subsubsection{xDSL}
xDSL \cite{xdsl-home} is a framework designed to simplify the creation and manipulation of domain-specific languages (DSLs) used in intermediate representations. It is written in Python, which results in lower barriers to entry and faster development than equivalent IRs written in lower level languages. It provides a set of constructs that can be used to define the syntax and semantics of a DSL, thus allowing the compiler or tool writer to focus on the logic of the transformation or analysis they're implementing rather than the details of parsing or code generation.

xDSL supports multiple extensions to its IR, called `Dialects', making it a versatile tool in the development of compilers and static analysers. It facilitates the optimization and analysis of the different dialects using shared algorithms and tools, and the ability to translate between dialects, which is especially important in the context of modern software stacks that employ multiple programming languages and target diverse execution environments.

The xDSL framework provides a set of capabilities that are useful for compiler infrastructure, the majority of which are modular and extensible. This modularity allows components to be reused across different projects and promotes an open-source development model where individual components can be developed independently. It supports the design of compiler pipelines which promote a clean separation between IR design and optimization algorithms.

\section{Ethical Considerations}
At this project is on developing a system design tool, there is a non-zero chance, as with any freely available tools in this field, that the project may be used in military applications by any state or by bad actors to produce hardware with the intent of subverting organisations and governments.

As attempting to introduce measures to prevent military use would likely have a detrimental effect on the functionality of the tool for regular users, and bad actors using freely available software to commit crimes does not change the illegality of these crimes.

When considering the direct ethical impacts of the tools created and presented in this paper, there is nothing notable to report as no personal data is being used and using more efficient hardware will reduce power consumption and therefore greenhouse gas emissions.

It is also useful to note that all code used in this project will be released on GitHub at the conclusion of this project in the \href{https://github.com/1rre/FlowDSL}{1rre/FlowDSL} repository. This will provide further resources to those who wish to use xDSL for hardware or to investigate static streaming dataflow programming for hardware.

\par\noindent\hrulefill\par

In summary, this chapter has provided the foundational knowledge required to understand the theoretical underpinnings of Flow DSL, with regards to High Level Synthesis, the Dataflow Programming Model, Embedded Domain-Specific Languages, and Intermediate Representation Frameworks. Together, these areas encompass the main components and concepts that form the basis of Flow DSL and its operation.

We also delved into the ethical implications of this work, acknowledging the potential for misuse while emphasizing the project's primary purpose as a system design tool meant to enhance productivity and efficiency in hardware design.
% Implementation
\chapter{Requirements}\label{chap:requirements}
In this chapter we discuss the research and technical requirements of this project, in terms of high level, language design and performance requirements.

\section{Research Question}
The main research question to be answered by this project is ``What reaching applications could static streaming dataflow optimisations have for hardware compilation and compilation in general?'' This question should be answered by evaluating the type of project or design which benefits from being analysed from a static streaming dataflow viewpoint.

\section{Principles of Language Design}
The primary objective of the Flow DSL is to capture and represent dataflow graphs. These graphs are essential for modeling computations in a way that is inherently parallel and can be efficiently mapped to hardware, such as FPGAs. The language design is centreed around the principles of simplicity, expressiveness, and integration with Python. These principles guide the development of features and abstractions in the language.

Flow DSL aims to provide a simple and intuitive syntax for defining dataflow graphs. By leveraging Python's syntax, the language ensures that users familiar with Python can easily adapt to the DSL. The simplicity principle should be reflected in the minimalistic approach to defining dataflow graph.

While simplicity is key, the language must also be expressive enough to capture complex dataflow graphs. This includes support for various data types and operations. The language provides abstractions for nodes and edges, and supports a range of arithmetic and logical operations. There should also be support for cycles within the dataflow graph.

\section{High Level Requirements}
The envisioned DSL should provide an interface for defining input and output streams, facilitating communication with external data sources. These external sources are beyond the control of the hardware, necessitating the assumption of a fixed one value per input stream per clock cycle. This requirement, in turn, leads to the necessity for a reset functionality capable of performing within a single clock cycle, enabling the system to reset in between batch computations and maintain continuous operation.

The DSL should offer a time-agnostic way of declaring a dataflow graph in a high level language for compilation to hardware. Such a graph should be expressed in terms of its inputs, outputs, and the operations that transform data from the former to the latter. This approach allows users to focus on the logical sequence of data transformation without being concerned about the temporal aspects of the computation, while allowing the compiler to retime and relocate parts of the graph without violating the constraints set by the user.

As stencil computations, which involve computations over multidimensional grids and are common in applications such as image processing, and finite impulse response (FIR) filters, are a frequent use of static synchronous streaming hardware (SSSH), operations common in these filters such as multiplication, summation, and delays should be supported.

The language should also be able to serve as an intermediate representation in its own right, capturing the semantics of an HLS hardware design in a dataflow graph which is far more amenable to optimisation and methods such as formal verification to ensure the correctness of the compilation process.

\section{Dataflow}
When considering static streaming dataflow graphs, there are a number of abstractions from both traditional control flow based imperative programming models and from RTL languages for developing hardware that must be kept, where possible, in a DSL designed to allow for the creation of these graphs. The structure of the graph is specified, however the timing is not, therefore optimisations can be applied by retiming and adding or removing buffers in different parts of the graph to encourage reuse and efficient computation.

\section{Streaming}
In a Static Streaming Dataflow Graph (SSDG), inputs are assumed to be populated and outputs extracted at regular intervals. By use of clock dividers and counters, the input can be set as the baseline at one input per clock cycle. While it is theoretically possible to have a situation in an SSDG where the output rate is faster than the input rate, as well as shown in figure \ref{req.streaming.out}, this is deemed to be out of scope for this project. It is still possible to implement, as each input can simply be fed in twice, however no specific support or optimisations for this will be added. There is also a possibility for a situation where the input rate is faster than the output rate, however in this case half of the inputs can just be ignored at the cost of potentially missing out on optimisations resulting in inputs not needing to be calculated every clock cycle. By establishing that the graph is always active as an input arrives on every update and that every output is used, we remove the need for any control flow within the DSL.

\begin{figure}[H]
  \centering
\begin{tikzpicture}
  
  \edef\sizetape{0.7cm}
  \tikzstyle{istream}=[draw,minimum height=\sizetape,minimum width=2 * \sizetape]
  \tikzstyle{iactive}=[arrow box,draw,minimum size=.5cm,arrow box arrows={east:0.25cm}]
  
  \begin{scope}[start chain=1 going right,node distance=-0.15mm]
      \node [on chain=1,istream] (in0) {$i_0$};
      \node [on chain=1,istream] (in1) {$i_1$};
      \node [on chain=1,istream] (in2) {$i_2$};
  \end{scope}
  \node [iactive,yshift=-.3cm,minimum width=2*\sizetape] at (in1.south) (ihead) {$in$};

  \tikzstyle{ostream}=[draw,minimum size=\sizetape]
  \begin{scope}[start chain=2 going right,node distance=-0.15mm,below=of ihead]
    \node [on chain=2,ostream,below=of in0,yshift=-6cm,xshift=-0.35cm] (out0) {$o_0$};
    \node [on chain=2,ostream] (out0a) {$o_1$};
    \node [on chain=2,ostream] (out1) {$o_2$};
    \node [on chain=2,ostream] (out1a) {$o_3$};
    \node [on chain=2,ostream] (out2) {$o_4$};
    \node [on chain=2,ostream] (out2a) {$o_5$};
  \end{scope}
  \node [iactive,yshift=.3cm] at (out0a.north) (ohead) {$out$};
      

  \begin{scope}
    \node (buf_0) [buf,below left=of ihead,yshift=0.5cm,xshift=0.6cm]{last};
    \node (add_1) [add,below=of buf_0,xshift=0.5cm,yshift=0.5cm];
    \node (div_1) [div,below=of add_1,yshift=0.5cm];
    \draw[->](ihead)--(buf_0);
    \draw[->](ihead)--(add_1);
    \draw[->](buf_0)--(add_1);
    \draw[->](add_1)--(div_1);
    \draw[->](div_1)--(ohead);
  \end{scope}
  
  \end{tikzpicture}
  \caption{An SSDG where the output is clocked faster than the input}\label{req.streaming.out}
\end{figure}

\par\noindent\hrulefill\par

In this section we detailed the research and technical requirements for this project, along with detailing some related aspects which are deemed to be out of scope of the project.

\chapter{Language Design}\label{chap:embedding}

This section discusses the specific design of the domain-specific language within Python and the method for embedding the DSL within Python and the decision making process behind this.

\section{Introduction}

\section{Types}
\subsection{Graph}
A graph represents the largest unit of compilation, equivalent to a module in a HDL or a full program in a software programming language. A single graph is instantiated and used as a base, similar to the \lstinline|Sequential()| class in Keras.

\subsection{Node}
A node in the Flow DSL differs slightly to a node in a dataflow graph insofar as a node in the Flow DSL not only can represent some form of operation on the data, it can also be a no-op to give information to the compiler, or used as operands for other nodes, implicitly getting its result edge, rather than explicitly calling a \lstinline|Node.result()| function.

\subsection{Data Types}
Currently all values are treated as unsigned integers, the only exceptions being for the \lstinline|-| and \lstinline|abs| unary operations and the subtrahend in the binary \lstinline|-| operation. The inputs to these operations are signed in all cases due to their inherently sign sensitive nature, however their results are not treated as signed. It is possible to have an integral input, output or const of any given width, however the default width is 32 bits. The widths of all other nodes are decided on context, as any bits which will not have an effect on any output will be eliminated.

\section{Operators and Functions}
\subsection{Logical and Arithmetic operators}
The binary logical and arithmetic operators, ie \lstinline|*|, \lstinline|/|, \lstinline|%|, \lstinline|+|, \lstinline|-|, \lstinline|&|, \lstinline||| and \lstinline|^| for binary operators all perform the same role as they do in Python, and are all implemented with the `BinOp' operation. While it would be possible to implement a different IRDL operation for each of these binary operators, which would additionally make optimisations based on redundant logical operations simpler, this would be a future expansion and was not done to reduce the complexity of the dialects.

Each binary operator has a result width, this is based on the width of its inputs and/or the used width of its outputs. Once processed, the resultant truncated inputs and outputs are as stated in the table \ref{embedding.binops.width}. As all values are treated as unsigned integers, there is zero extension rather than sign extension applied to either operand in the case of mismatched widths.

\subsubsection{Slices}
In many cases, in hardware it may be necessary to perform operations on only part of a wire, for example when processing RGBA pixel data or divider output in a \lstinline|{hi, lo}| format. While it may be possible to truncate the least significant bits of a value using a right shift, and the most significant bits using a bitwise and with a constant, this is a lot of work for such a simple operation. Slices work as they do in Python, ie with the square bracket operator, for instance \lstinline|Node[int:int]|. One can also perform a slice of a single bit with simply \lstinline|Node[int]|, which implicitly sets the upper bound of the slice to one greater than the lower bound, which is set to the operand.

\subsubsection{Concatenation}
As with slices, it is often necessary to merge multiple values together, either into a bus or into a larger single wire \hyphen{} often as the inverse of slicing. Again, this is theoretically possible with a combination of shifts and binary operations, however it is again a convoluted process for such a common operation. We therefore provide an additional binary operator, the \lstinline|@| operator, which is used for concatenating the value of two output edges. This operator was chosen as it is already used for concatenation in languages such as F\#.

\begin{table}[h]
  \centering
  \begin{tabular}{|c|c|c|c|}
    \hline
    \textbf{Operator} & \textbf{Input 1 Width} & \textbf{Input 2 Width} & \textbf{Output Width} \\
    \hline \lstinline|&| (BAND) & $w_1$ & $w_2$ & $\max(w_1, w_2)$ \\
    \hline \lstinline||| (BOR) & $w_1$ & $w_2$ & $\max(w_1, w_2)$ \\
    \hline \lstinline|^| (XOR) & $w_1$ & $w_2$ & $\max(w_1, w_2)$ \\
    \hline \lstinline|+| (ADD) & $w_1$ & $w_2$ & $\max(w_1, w_2) + 1$ \\
    \hline \lstinline|-| (SUB) & $w_1$ & $w_2$ & $\max(w_1, w_2) + 1$ \\
    \hline \lstinline|*| (MUL) & $w_1$ & $w_2$ & $w_1 + w_2$ \\
    \hline \lstinline|/| (DIV) & $w_1$ & $w_2$ & $w_1$ \\
    \hline \lstinline|%| (MOD) & $w_1$ & $w_2$ & $w_1$ \\
    \hline \lstinline|<<| (LSL) & $w_1$ & $c_1$ & $w_1 + c_1$ \\
    \hline \lstinline|>>| (LSR) & $w_1$ & $c_1$ & $w_1 - c_1$ \\
    \hline \lstinline|and| (LAND) & $w_1$ & $w_2$ & $1$ \\
    \hline \lstinline|or| (LOR) & $w_1$ & $w_2$ & $1$ \\
    \hline \lstinline|==| (EQ) & $w_1$ & $w_2$ & $1$ \\
    \hline \lstinline|!=| (NEQ) & $w_1$ & $w_2$ & $1$ \\
    \hline \lstinline|>=| (GE) & $w_1$ & $w_2$ & $1$ \\
    \hline \lstinline|>| (GT) & $w_1$ & $w_2$ & $1$ \\
    \hline \lstinline|<=| (LE) & $w_1$ & $w_2$ & $1$ \\
    \hline \lstinline|<| (LT) & $w_1$ & $w_2$ & $1$ \\
    \hline \lstinline|@| (CAT) & $w_1$ & $w_2$ & $w_1 + w_2$ \\
    \hline \lstinline|[c1:c2]| (SLICE) & $w_1$ & $c_1, c_2$ & $c_2 - c_1$ \\
    \hline
  \end{tabular}
  \caption{Default bit width of binary operators based on input widths}\label{embedding.binops.width}
\end{table}

\subsubsection{Unary Operators}
The unary operators in the Flow DSL are \lstinline|not|, \lstinline|~|, \lstinline|-| and \lstinline|+|. Unary \lstinline|not|, \lstinline|~| and \lstinline|-| perform the same role one would expect, that is logical not, bitwise not and negation respectively. The unary \lstinline|+| operator works as a resettable accumulator, that is an accumulator which has the current value treated as zero if the reset signal is held low. This functionality is also accessible through \lstinline|Graph.accumulate(node)|, however as accumulators are so commonly used a unary operator was dedicated to them.

\subsubsection{Resets}
Similarly to the accumulator, users will be able to generate their own single cycle accumulator, or have a general reset signal, using the \lstinline|Node.with_reset(int)| function. This function will result in a wire, as opposed to a register, which will mirror the output of the node it is called on unless \lstinline|reset_n|, an implicitly introduced signal in the hardware design, is high. The accumulator, \lstinline|Graph.accumulate(node)|, is a subset of a resettable signal, however due to how frequently addition accumulators are used, it was given its own function.

\subsection{Offsets}
Offsets are possible using the \lstinline|Node.offset(int)| function, gets the value of the output edge of that node in \lstinline|n| clock cycles if \lstinline|n| is positive, or the value of the output edge \lstinline|-n| clock cycles ago if \lstinline|n| is negative, where \lstinline|n| is the operand to the offset function. While it would have been possible to implement \lstinline|Node.last()| and \lstinline|Node.next()|, or alternatively \lstinline|Node.last(int)| and \lstinline|Node.next(int)|, this would've required extra processing to deal with negative inputs to these functions anyway, therefore it was deemed simpler and equal in functionality to use a single function for both offsets into the future and into the past. The ability to access the previous and future values of an edge are integral to dataflow computing, as they allow for running totals akin to the \lstinline|scan| function in many functional programming languages, as can be seen in figure \ref{design.mvg_avg.total}. \lstinline|fold| and \lstinline|reduce| are also implicitly possible using a top level module \hyphen{} to acheive this one would use a reset signal to reset the accumulator when input begins, and simply ignore the output until all input data is processed.

\begin{figure}[H]
  \centering
  \begin{tikzpicture}[node distance=1cm,circuit ee IEC]
    % below n
    \node (n) {$[x_{-n}, ..., x_{-1}, x_0, x_1, ..., x_{n}]$};
    \node (buf_0) [buf,below left=of n,xshift=1.5cm]{last};
    \node (buf_1) [buf,below=of buf_0]{last};
    \node (buf_2) [buf,below=of buf_1]{last};
    \node (add_0) [add, below=of n,yshift=-1cm];
    \node (buf_3) [buf,below right=of add_0]{last};
    \node (sub_0) [sub, below left=of buf_3];
    \node (div_0) [div_4,below=of sub_0];
    \node (y) [below=of div_0] {$[y_{-n}, ..., y_{-1}, y_0, y_1, ..., y_{n}]$};
    \draw[->](n)--(buf_0);
    \draw[->](buf_0)--(buf_1);
    \draw[->](buf_1)--(buf_2);
    \draw[->](n)--(add_0);
    \draw[->](buf_3)--(add_0);
    \draw[->](add_0)--(sub_0);
    \draw[->](buf_2)--(sub_0);
    \draw[->](sub_0)--(div_0);
    \draw[->](sub_0)--(buf_3);
    \draw[->](div_0)--(y);
  \end{tikzpicture}
  \caption{Cyclic Dataflow Graph for a Moving Average}\label{design.mvg_avg.total}
\end{figure}

\subsection{Forward Referencing}
Due to the time-sensitive graph-based nature of the Flow DSL, a variable does not need to be instantiated in code for it to be instantiated in the graph. An example of this is the moving average calculator shown in figure \ref{design.mvg_avg.total}. The same is true for any possible cyclic graph; so long as the value of an edge doesn't depend on its value at the current time, it is a valid dataflow graph and can be compiled to hardware via the Flow DSL. If the value of an edge is found to depend on its value at the current, a compiler error is produced as the graph is unrepresentable in hardware. In the figure, the \lstinline|last| nodes take the previous value of the vertex going into them. A \lstinline|next| node would take the next value of the vertex going into them, however as the next value is unknown at a given time, it can and must be rewritten as a \lstinline|last| node; see chapter \ref{chap:compilation} for more information on this.

\section{Variable Naming}
Variable names are able to be used much in the same way as Python. This includes the overwriting of variables \hyphen{} the code samples in listings \ref{python.varname.loop} and \ref{python.varname.array} are functionally identical.

\renewcommand\theFancyVerbLine{\arabic{FancyVerbLine}}
\begin{listing}[H]
  \begin{minted}[numbers=left]{python3}
named = gr.const(0)
for i in range(1, 5):
  named = in0 + named
out_named = gr.ostream(in0 // named)
  \end{minted}
  \caption{Overwriting variables using renaming}\label{python.varname.loop}
\end{listing}

\renewcommand\theFancyVerbLine{\arabic{FancyVerbLine}}
\begin{listing}[H]
  \begin{minted}[numbers=left]{python3}
named = [gr.const(0)]
for i in range(1, 5):
  named.append(in0 + named[-1])
out_named = gr.ostream(in0 // named[-1])
  \end{minted}
  \caption{Keeping variables live in Python using a list}\label{python.varname.array}
\end{listing}

\section{Alternatives Considered}

In the development of the Flow DSL, several alternatives and approaches were considered to handle variable referencing and attribute access within the graph. Two notable concepts that were considered were the so-called `addvar' and `getvar' methods, and the use of Python's \lstinline|__getattr__| and \lstinline|__setattr__| magic methods. We will discuss each of these approaches and analyze their potential advantages and disadvantages.

\subsection{Using addvar and getvar for Variable Referencing}
The `addvar' and `getvar' concept was used to reference variables based on their names. The `addvar' method would be used to add a variable to the graph, while the `getvar' method would be used to retrieve the value of a variable from the graph using its name. This approach was not used due to a number of drawbacks \hyphen{} firstly, it was incompatible with existing linters, unlike non-forward references in the final version. This negates some of the benefits of using an embedded DSL. Additionally the variable names used in Python were not the same as those emitted in Verilog. There were also the same function scoping issues as described in section \ref{varname}, however as the user would be using strings, rather than variable names, the disconnect is even more obvious.

\subsection{Using \texttt{\textunderscore\textunderscore getattr\textunderscore\textunderscore} and \texttt{\textunderscore\textunderscore setattr\textunderscore\textunderscore} Magic Methods}
An different approach to variable naming is to use Python's \lstinline|__getattr__| and \lstinline|__setattr__| magic methods. These methods are automatically called when getting or setting an attribute of an object, respectively. These could be called on the \lstinline|Graph| object, rather than using the varname package, with all other features being the same. This would likely be a better alternative to using varname, as varname is fairly unstable and relies on Python internals whereas \lstinline|__getattr__| and \lstinline|__setattr__| are built-in methods. The one issue with \lstinline|__getattr__| and \lstinline|__setattr__| is that they may interract with methods and objects within the graph in unexpected ways, especially if a variable name happens to share a name with one of these.

\par\noindent\hrulefill\par

This section discussed the design and methodology behind Flow DSL's embedding in Python, and the rationale behind that design and methodology. Graph and Node types were introduced, along with the operations which work on these types.
\chapter{Implementation}\label{chap:compilation}
THis section aims to introduce and explain the implementation details of Flow DSL and the motivation behind this.

\section{Introduction}
Flow DSL is implemented in Python atop xDSL. While at time a strong and powerful type system such as that in Scala 3, OCaml or F\# would have been useful \hyphen{} there are a lot of uses of \lstinline|isinstance| which would've been more pleasant to deal with given a stronger type system and plenty of `errors' from PyLint due to its lack of recognition of \lstinline|hasattr|, as well as the lack of polymorphism, however at other times it was incredibly useful to have the simplicity and modularity of Python. The ability to leverage xDSL was also incredibly valuable \hyphen{} if the project were to be done in Scala 3, OCaml or F\# it is likely that either no IR framework or an inferior one would've been needed to be used.

\section{Design Features}

\subsection{Variable Naming}\label{varname}
The Python \lstinline|varname| package is used to capture Python variable names with the intention of improving the readability of the generated HDL code. Using \lstinline|varname| in combination with \lstinline|Graph.forward_ref(str)| also allows for code to be written in an intuitive way, similar to how one would write Python for interpretation, rather than having to access variables with a string ID each time. Having \lstinline|Graph.forward_ref(str)| as the only user-facing function to deal with a variable by a string name, rather than a direct reference, minimises the confusion which may occur when a user shadows a variable or overwrites it within Python, as in as in listing \ref{python.varname.loop}. As Python's mutability of variables fundamentally contradicts the expectation of immutability of a named variable in Verilog and other single-assignment languages, a workaround had to be created for the case where the user were to assign multiple variables to the same name. While this requires a workaround on the backend, it is worth noting that this is more compatible with Python than enforcing single assignment to a name. An example of this is a for loop where the user is creating a pipeline where the same operation is applied iteratively to a given value. While a functionally identical version of the method could be achieved \lstinline|list.append| with the added benefit of variables being accessible outside of the iterative process, this is far less intuitive as a construct within the language.

\section{xDSL Dialects}
The DSL backend, which is built atop xDSL, consists of four dialects. These are, in order of invocation, \lstinline|flow_initial|, \lstinline|flow2|, \lstinline|flow_timed| \& \lstinline|hard_flow|. Each layer decended by advancing a dialect adds information to take the program closer to the register transfer level.

\section{Compiler Passes}
\subsection{Graph to xDSL}\label{graph.2.xdsl}
The first compilation pass is not implemented with xDSL's \lstinline|ModulePass| class as the representation uses simply Python classes, extracted as they are declared using the \lstinline|varname| package and a map to keep track of nodes to names and references. `Anonymous', or unnamed, variables, such as \lstinline|y.offset(1)| in \lstinline|x + y.offset(1)| are given a random unique identifier according to Python's \lstinline|uuid| module.

While it would be possible to write the graph definitions such that the initial version of the xDSL dialect is generated directly when the interpreter executes the line which declares the code, using a decoupled front end leaves options for different backends to be added in the future. This also allows for easier control over the names and forward references, as such operations are not well suited for xDSL's \lstinline|RewritePattern| class to handle. Even then, however, the \lstinline|flow_initial| does not reference by value, it instead references by name \hyphen{} no results of computations are used. This means that compiler optimisation passes are not easy to implement in this stage.

At this point bit widths are only assigned to the \lstinline|IStream|, \lstinline|OStream|, \lstinline|slice| and \lstinline|Const| operations and their results using the \lstinline|flow_initial.wnode| attribute, rather than the \lstinline|flow_initial.node| attribute which does not not carry a width. An example of the \lstinline|flow_initial| dialect is shown in listing \ref{dialect.flow_initial}.

\renewcommand\theFancyVerbLine{\arabic{FancyVerbLine}}
\makeatletter
\AtBeginEnvironment{minted}{\dontdofcolorbox}
\def\dontdofcolorbox{\renewcommand\fcolorbox[4][]{##4}}
\makeatother
\begin{listing}[H]
  \begin{minted}[numbers=left, breaklines]{llvm}
%0 = "flow_initial.istream"() : () -> !flow_initial.wnode<#int<32>, "in0">
%1 = "flow_initial.istream"() : () -> !flow_initial.wnode<#int<32>, "in1">
%2 = "flow_initial.binop"() {"op" = "*", "left" = "in0", "right" = "in1"} : () -> !flow_initial.node<"current">
%3 = "flow_initial.binop"() {"op" = "+", "left" = "#1#", "right" = "current"} : () -> !flow_initial.node<"sum_out">
%4 = "flow_initial.offset"() {"node" = "sum_out", "offset" = #int<-1>} : () -> !flow_initial.node<"#1#">
%5 = "flow_initial.ostream"() {"node" = "sum_out"} : () -> !flow_initial.wnode<#int<32>, "out0">  
  \end{minted}
  \cprotect\caption{An implementation of an inner product in the \lstinline|flow_initial| dialect}
  \label{dialect.flow_initial}
\end{listing}

\subsection{Reference Matching}
To convert the operations within the initial dialect from referencing by a variable by its string name to referencing its xDSL \lstinline|SSAValue| (Single Static Assignment Value) a new dialect, \lstinline|flow2|, which is similar in most ways to \lstinline|flow_initial| is used, however references other nodes by their SSAValues. As can be seen in listing \ref{dialect.flow_2}, xDSL's printer does not expect forward references, and therefore does not assign a SSAValue to each result before starting printing, instead assigning SSAValues during printing. This means that forward references appear to xDSL as use of results which are not added to the IR. While it would be possible to fix this, the xDSL printer is only used for debugging and an entirely new printer was defined in section \ref{sect:compiling.flow_printer}. An indication of the above program with properly referenced \lstinline|SSAValue|s can be seen below in listing \ref{dialect.flow_2.fixed}, and in all future xDSL snippets, however note that these have been modified from the verbatim output only to remove false positives on undeclared variables.

Only non-generated names are kept between \lstinline|flow_initial| and \lstinline|flow2| \hyphen{} this is because nodes may be added or removed during the compilation process, therefore names must be created for unknown nodes at the end anyway, and it makes sense to have these be consistent throughout.

\makeatletter
\AtBeginEnvironment{minted}{\dontdofcolorbox}
\def\dontdofcolorbox{\renewcommand\fcolorbox[4][]{##4}}
\makeatother
\begin{listing}[H]
  \begin{minted}[numbers=left, breaklines]{llvm}
%0 = "flow2.istream"() {"uid" = "in0"} : () -> !flow2.wnode<#int<32>>
%1 = "flow2.istream"() {"uid" = "in1"} : () -> !flow2.wnode<#int<32>>
%2 = "flow2.binop"(%0, %1) {"op" = "*", "uid" = "current"} : (!flow2.wnode<#int<32>>, !flow2.wnode<#int<32>>) -> !flow2.node
%3 = "flow2.binop"(%2, %<UNKNOWN>) {"op" = "+", "uid" = "sum_out"} : (!flow2.node, !flow2.node) -> !flow2.node
-----------------------^^^^^^^^^^----------------------------------------------------------------
| ERROR: SSAValue is not part of the IR, are you sure all operations are added before their uses?
-------------------------------------------------------------------------------------------------
%4 = "flow2.offset"(%3) {"offset" = #int<-1>} : (!flow2.node) -> !flow2.node
%5 = "flow2.ostream"(%3) {"uid" = "out0"} : (!flow2.node) -> !flow2.wnode<#int<32>>wnode<#int<32>>
  \end{minted}
  \cprotect\caption{An implementation of an inner product in the \lstinline|flow_2| dialect}\label{dialect.flow_2}
\end{listing}

\makeatletter
\AtBeginEnvironment{minted}{\dontdofcolorbox}
\def\dontdofcolorbox{\renewcommand\fcolorbox[4][]{##4}}
\makeatother
\begin{listing}[H]
  \begin{minted}[numbers=left, breaklines]{llvm}
%0 = "flow2.istream"() {"uid" = "in0"} : () -> !flow2.wnode<#int<32>>
%1 = "flow2.istream"() {"uid" = "in1"} : () -> !flow2.wnode<#int<32>>
%2 = "flow2.binop"(%0, %1) {"op" = "*", "uid" = "current"} : (!flow2.wnode<#int<32>>, !flow2.wnode<#int<32>>) -> !flow2.node
%3 = "flow2.binop"(%2, %4) {"op" = "+", "uid" = "sum_out"} : (!flow2.node, !flow2.node) -> !flow2.node
%4 = "flow2.offset"(%3) {"offset" = #int<-1>} : (!flow2.node) -> !flow2.node
%5 = "flow2.ostream"(%3) {"uid" = "out0"} : (!flow2.node) -> !flow2.wnode<#int<32>>
  \end{minted}
  \cprotect\caption{An implementation of an inner product in the \lstinline|flow_2| dialect, with errors fixed.}\label{dialect.flow_2.fixed}
\end{listing}

\subsubsection{Checking Names}\label{dialect.namecheck}
To form the \lstinline|flow2| dialect from \lstinline|flow_initial|, the nodes referenced by name must be matched to the nodes which declare them. This is the role of the \lstinline|ReferenceMatching| `module pass' and the \lstinline|CheckNames| `rewrite patter'. Due to the potential existence of forward references, a dictionary mapping known names is passed to \lstinline|CheckNames| along with a list of nodes currently in the graph.

A challenge is presented here by the xDSL framework specifying that a ``rewrite pattern' should be “A side-effect free rewrite pattern matching on a DAG.'' While a truly side-effect free implementation would theoretically be possible by storing the \lstinline|ModuleOp| object which encapsulates all active nodes in the dataflow graph, and searching though each of the nodes within the module for the one with the name which matched the current operand, this would both require keeping temporary names for longer than needed \hyphen{} why this is undesirable is covered in section \ref{naming}, and be wildly inefficient as it would require a linear search for each operand in the graph, likely multiple times given all operands of a node must have already been converted to a `Flow 2 Node' for the node itself to be converted. There is also the consideration that cyclic dataflow graphs fundamentally breach the requirement for the xDSL representation of the program to be a DAG (Directed Acyclic Graph) \hyphen{} this is simply not possible given the fact that cyclic dataflow graphs are by definition not DAGs. Given one of the conditions \hyphen{} and the condition which is more liable to prove an issue due to future changes to xDSL's internals at that, is being ignored, it is not unreasonable to ignore the condition for a lack of side-effects, so long as the side-effects don't reach beyond the internal data of the rewrite pattern class.

To rewrite the \lstinline|flow_initial| dialect, therefore, a dictionary of node names to nodes is created. Using the \lstinline|PatternRewriteWalker| class, each operation within the graph is iterated, with any instances of \lstinline|flow_initial| nodes being rewritten as the equivalent \lstinline|flow2| nodes. The new \lstinline|flow2| nodes are added to the dictionary to replace the old \lstinline|flow_initial| nodes, as the nodes in the dictionary are used to resolve operands, which may be forward or backward references. xDSL currently handles forward references without much extra external effort to ensure this, however as it is an out-of-spec usage there are no guarantees this will continue in the future. Speaking with the xDSL team has confirmed that there is not a better way to do this other than creating a separate rewrite system which explicitly supports forward references and side effects, at which point it likely would have been just as useful to create an IR framework from scratch or use a different framework.

\subsubsection{Dead Code Eliminator}
The `Dead Code Eliminator' (DCE) module removes code which is inaccessible, or `dead'. In the context of dataflow programming, where there is no control flow to stop the propagation of values into certain paths, this means any nodes which aren't connected to both at least one input node and at least one output node. Without being connected to an input node or a constant value, there is no way for the nodes to get values \hyphen{} in Verilog these would result in a net of \lstinline|X| (uninitialised) or \lstinline|Z| (undriven) bits. If a node is not connected to any output nodes there is no way for its value to affect the output, therefore if we treat the dataflow graph as a black box, the node may as well not be there at all. While there are special cases, such as binary operations which only have a single operand, any node with at least one connection to an input and one connection to an output is considered `live' for the purposes of this project.

Checking whether code is `dead' is more challenging in cyclic dataflow graphs than in software or acyclic dataflow graphs due to the existence of cycles. This means that the result of an operation will be used in the computation for that same operation at a later time, therefore a naïve dead code detector would identify that node as used, or worse, enter an infinite loop constantly checking the liveness of the cycle. To resolve this, the algorithms \ref{algo.dce.out}, for output nodes, and \ref{algo.dce.in}, for input nodes, are used. An intersection is then taken of the two sets to determine which nodes are `live'. Any nodes which are not `live' are removed from the graph.

As xDSL expects graphs to be acyclic, attempting to remove a cyclic graph can present an issue. Nodes which use themselves as an operand cannot be deleted, as the protections in xDSL prevent the erasure of nodes which are still used, therefore the \lstinline|safe_erase| flag on the erasure must be set to false, allowing the node to be erased even if it still has uses. As all uses of the node will also be `dead' if the node itself is `dead', this is not an issue as all those nodes will be deleted by the DCE algorithm in the current cycle. As the \lstinline|drop_references| flag is set to true, all operations `upstream' of the node being erased will have the node removed as a use, even though the node is not safely deleted, therefore there will be no issues with nodes having `phantom' uses in later stages.


\renewcommand\theFancyVerbLine{\arabic{FancyVerbLine}}
\begin{algorithm}[H]
\caption{Dead Code Elimination: Collecting Outputs}\label{algo.dce.out}
\begin{algorithmic}[1]
\Function{AddDepsOut}{$op <: $ Flow2Node, $deps <: $ set[Flow2Node]}
\If{$op \notin deps$}
\State $deps \gets deps \cup op$
\For{$next \gets op.operands$}
\State \Call{AddDepsOut}{$next$, $deps$}
\EndFor
\EndIf
\EndFunction
\Function{GetOutputDeps}{$module <: $ ModuleOp} $ \rightarrow $ set[Flow2Node]
\State $map \gets \emptyset$
\For{$op \gets module.ops$}
\If{$op <: $ OStream}
\State \Call{AddDepsOut}{$op$, $map$}
\EndIf
\EndFor
\Return $map$
\EndFunction
\end{algorithmic}
\end{algorithm}

\renewcommand\theFancyVerbLine{\arabic{FancyVerbLine}}
\begin{algorithm}[H]
\caption{Dead Code Elimination: Collecting Inputs}\label{algo.dce.in}
\begin{algorithmic}[1]
\Function{AddDepsIn}{$op <: $ Flow2Node, $deps <: $ set[Flow2Node]}
\If{$op \notin deps$}
\State $deps \gets deps \cup op$
\For{$next \gets op.uses$}
\State \Call{AddDepsIn}{$next$, $deps$}
\EndFor
\EndIf
\EndFunction
\Function{GetInputDeps}{$module <: $ ModuleOp} $ \rightarrow $ set[Flow2Node]
\State $map \gets \emptyset$
\For{$op \gets module.ops$}
\If{$op <: $ Const $ \vee $ IStream}
\State \Call{AddDepsIn}{$op$, $map$}
\EndIf
\EndFor
\Return $map$
\EndFunction
\end{algorithmic}
\end{algorithm}

\subsection{Flow2 Optimisations}
The `Deduplicator' module pass works on the \lstinline|flow_2| xDSL dialect and performs the bulk of the compiler optimisations. This is because optimisation after timing has been added to the system would result in the system being out of sync or having additional buffering. Few further optimisations could be made possible as a result of introducing timing \hyphen{} most of these would be relating to mutually exclusive bits, which is an optimisation which is not carried out anyway due to the expectation of all nodes of one input per clock cycle. This module pass consists of an `Offset Renamer', `Offset Replacer' and `Expression Eliminator'. As many of the rewrite patterns in this module pass can uncover more optimisations for another rewrite pattern, for instance the `Offset Replacer' may move an offset across a binary operation, allowing the `Offset Renamer' to merge it with another offset which was applied to the binary operation, each of the patterns are applied sequentially until none of them make any changes. While this would lead to an infinite loop if the rewrite patterns repeatedly made and undid each others' changes, for any well defined input the graph will settle and the loop will terminate.

\subsubsection{Offset Renamer}
Consider the code in listing \ref{optimisation.offset_rm} in the Flow DSL. This code repeatedly applies offsets of one clock cycle to an offset node, rather than offsetting from a non-buffer node \hyphen{} in this case the istream node. This can make cycle detection and timing harder by convoluting the xDSL \lstinline|uses| and \lstinline|operands| systems. By writing a rewrite rule which applies only to offsets which operate on an offset, any dual offsets can be eliminated to ensure all offsets apply only to a logic node, rather than buffering node. This will allow for the same buffers to be used for multiple offsets and for other optimisations to get a clearer view of the program topology.

\renewcommand\theFancyVerbLine{\arabic{FancyVerbLine}}
\begin{listing}[H]
  \begin{minted}[numbers=left]{python3}
in0 = gr.istream()
total = gr.const(0)
for i in range(avg_window):
  total = in0 + total
  in0 = in0.offset(1)
gr.ostream(total / gr.const(avg_window))
  \end{minted}
  \caption{Repeated application of offset1}\label{optimisation.offset_rm}
\end{listing}

\subsubsection{Offset Replacer}
In dataflow graphs, performing a binary operation sooner rather than later will reduce the number of buffers required in a circuit. Consider the expression \lstinline|x.offset(1) + y.offset(2)| visible in the dataflow graph in figure \ref{dataflow.offset_rename} \hyphen{} this requires three buffers and an adder. By replacing it with \lstinline|(x + y.offset(1)).offset(1)|, we can reduce the number of buffers to two. The algorithm for this replaces any instance of a binary operation with two inputs from `offset' nodes with a version adjusted such that at most one of the operands is offset, and the result of the binary operation is offset. This can be seen in algorithm \ref{optimisation.replacing_offset}.

\renewcommand\theFancyVerbLine{\arabic{FancyVerbLine}}
\begin{algorithm}[H]
\caption{Replacing Offsets}\label{optimisation.replacing_offset}
\begin{algorithmic}[1]
\Function{RewriteOffset}{$op <: $ BinOp[Offset, Offset]}
\State $offset_0 \gets op.left.offset$
\State $offset_1 \gets op.right.offset$
\If{$offset_0$ > $offset_1$}
\State $offset_{in} \gets offset_0 - offset_1$
\State $offset_{out} \gets offset_1$
\State $new\_offset \gets $Offset($op.left.parent$, $offset_{in}$)
\State $new\_op \gets $Offset(BinOp($new\_offset$, $op.right.parent$), $offset_{out}$)
\State replace($op$, $new\_op$)
\ElsIf{$offset_0$ < $offset_1$}
\State $offset_{in} \gets offset_1 - offset_0$
\State $offset_{out} \gets offset_0$
\State $new\_offset \gets $Offset($op.right.parent$, $offset_{in}$)
\State $new\_op \gets $Offset(BinOp($op.left.parent$, $new\_offset$), $offset_{out}$)
\State replace($op$, $new\_op$)
\Else
\State $offset_{out} \gets offset_0$
\State $new\_op \gets $Offset(BinOp($op.left.parent$, $op.right.parent$), $offset_{out}$)
\State replace($op$, $new\_op$)
\EndIf
\EndFunction
\end{algorithmic}
\end{algorithm}

\begin{figure}[h]
  \centering
  \begin{subfigure}[t]{0.5\textwidth}\centering
    \begin{tikzpicture}[node distance=1cm,circuit ee IEC]
      % below n
      \node (x) [xshift=0cm]{$x$};
      \node (y) [right=of x,xshift=0cm]{$y$};
      \node (buf_0) [buf,below=of x]{last};
      \node (buf_1) [buf,below=of y]{last};
      \node (buf_2) [buf,below=of buf_1]{last};
      \node (add_0) [op, below=of buf_0]{$+$};
      \node (out) [below=of add_0] {$out$};
      \draw[->](x)--(buf_0);
      \draw[->](y)--(buf_1);
      \draw[->](buf_1)--(buf_2);
      \draw[->](buf_2)--(add_0);
      \draw[->](buf_0)--(add_0);
      \draw[->](add_0)--(out);
    \end{tikzpicture}
    \caption{Before Replacement}
  \end{subfigure}%
  \begin{subfigure}[t]{0.5\textwidth}\centering
    \begin{tikzpicture}[node distance=1cm,circuit ee IEC]
      % below n
      \node (x) [xshift=0cm]{$x$};
      \node (y) [right=of x,xshift=0cm]{$y$};
      \node (buf_0) [buf,below=of y]{last};
      \node (add_0) [op, below=of x]{$+$};
      \node (buf_1) [buf,below=of add_0]{last};
      \node (out) [below=of buf_1] {$out$};
      \draw[->](x)--(add_0);
      \draw[->](y)--(buf_0);
      \draw[->](buf_0)--(add_0);
      \draw[->](add_0)--(buf_1);
      \draw[->](buf_1)--(out);
    \end{tikzpicture}
    \caption{After replacement}
  \end{subfigure}
  \caption{Offset replacement: dataflow graphs}\label{dataflow.offset_rename}
\end{figure}

\subsubsection{Expression Eliminator}
Following the application of the offset renamer and replacer, there is potential for duplicate expressions both in the form of offsets where the same node is offset by the same amount, constants with the same integer value, and in binary operations where the same binary operation is performed on the same two operands \hyphen{} including in the reverse order on commutative operations. Considering the \lstinline|x.offset(1) + y.offset(2)| example from the `Offset Replacer' \hyphen{} if we were to have another available expression \lstinline|x + y.offset(1)|, we would change the former to \lstinline|(x + y.offset(1)).offset(1)|, after which we could merge both \lstinline|y.offset(1)| expressions, and then finally merge the \lstinline|x + y.offset(1)| expressions to produce the same result as the unoptimised code, but with one adder and two buffers rather than the original two adders and four buffers \hyphen{} a resource usage decrease of 50\%.

\subsection{Node Timing}
Nodes in the \lstinline|flow_2| dialect are untimed. This means that retiming, offsets and buffering cannot be checked and applied. To enable these to happen a new, timed, dialect is introduced: \lstinline|flow_timed|. A sample of the \lstinline|flow_timed| dialect implementing an inner product can be seen in listing \ref{dialect.flow_timed}.

\renewcommand\theFancyVerbLine{\arabic{FancyVerbLine}}
\makeatletter
\AtBeginEnvironment{minted}{\dontdofcolorbox}
\def\dontdofcolorbox{\renewcommand\fcolorbox[4][]{##4}}
\makeatother
\begin{listing}[H]
  \begin{minted}[numbers=left, breaklines]{llvm}
%0 = "flow_timed.istream"() {"uid" = "in0"} : () -> !flow_timed.wnode<#int<0>, #int<32>>
%1 = "flow_timed.istream"() {"uid" = "in1"} : () -> !flow_timed.wnode<#int<0>, #int<32>>
%2 = "flow_timed.binop"(%0, %1) {"op" = "*", "uid" = "current"} : (!flow_timed.wnode<#int<0>, #int<32>>, !flow_timed.wnode<#int<0>, #int<32>>) -> !flow_timed.node<#int<1>>
%3 = "flow_timed.binop"(%2, %4) {"op" = "+", "uid" = "sum_out"} : (!flow_timed.node<#int<1>>, !flow_timed.node<#int<1>>) -> !flow_timed.node<#int<2>>
%4 = "flow_timed.offset"(%3) {"offset" = #int<-1>} : (!flow_timed.node<#int<2>>) -> !flow_timed.node<#int<1>>
%5 = "flow_timed.ostream"(%3) {"uid" = "out0"} : (!flow_timed.node<#int<2>>) -> !flow_timed.wnode<#int<2>, #int<32>>  
  \end{minted}
  \cprotect\caption{An implementation of an inner product in the \lstinline|flow_timed| dialect}\label{dialect.flow_timed}
\end{listing}

\subsubsection{Setting Times}
The time setting algorithm currently requires that all logic involving operations on a node require one, and only one, clock cycle. Although this is suboptimal for reasons described in section \ref{ext.retime}, it is sufficient for all dataflow graphs, including cyclic graphs, which have sufficient dependency distance for there to be one clock cycle for each of these operations. When the accumulator and resettable nodes are used in place of most offsets to the last value, this retimable subset is the vast majority of dataflow graphs.

The timing algorithm assigns each node a \lstinline|lo| and \lstinline|hi| time for when it can be ready \hyphen{} these can be \lstinline|None| if there is no upper or lower limit, such as upper limits in the case of outputs, or if its limits have not been calculated yet. The first stage is to set all timings on inputs to \lstinline|lo, hi = 0|. This is because all inputs are populated at time zero by definition. This timing is then propagated forward though the graph according to the timings in table \ref{fwd.timings}. While the \lstinline|hi| timings are not too useful currently other than for detecting timing violations, when the new timing algorithm as described in section \ref{ext.retime} is implemented these will be much more useful.

Setting times is an iterative process \hyphen{} in cases of cycles, it is evident that the node will have one fewer input to set its timings based on in the initial run, however after one iteration, that input should be populated by at least something. More complex cycles may require more iterations, for instance if a node is cycled back to multiple points. Nodes are only updated from the \lstinline|flow2| to \lstinline|flow_timed| dialect when the values of its lower and upper limits match and are not \lstinline|None|, with the exception of constants, which are untimed by nature. Leaving nodes which are untimed as \lstinline|flow2| after the timings have reached a stable solution means that nodes which are unable to be timed due to their cycle dependency distance being too low to be easily identified and reported as compiler errors to the user. It is worth noting that in some cases, cycles will cause the lower and upper bounds to increase on each cycle, for instance if there are no restrictions after the untimeable node. These are detected as their \lstinline|lo| and \lstinline|hi| values will increase consistently on each iteration \hyphen{} this consistent behaviour can be tracked and caught.

\begin{table}[h]
  \centering
  \begin{threeparttable}
  \begin{tabular}{|c|c|c|c|}
    \hline
    \textbf{Node Type} & \textbf{Input 1} & \textbf{Input 2} & \textbf{Output} \\
    \hline IStream & \hyphen{} & \hyphen{} & $(0, 0)$ \\
    \hline OStream & $(t_{lo}, t_{hi})$ & \hyphen{} & $(t_{lo}, t_{hi})$ \\
    \hline BinOp & $(a_{lo}, a_{hi})$ & $(b_{lo}, b_{hi})$ & $(\max(a_{lo}, b_{lo}), \max(a_{hi}, b_{hi}) + 1)$ \\
    \hline UnaryOp & $(t_{lo}, t_{hi})$ & \hyphen{} & $(t_{lo} + 1, t_{hi} + 1)$ \\
    \hline Const & \hyphen{} & \hyphen{} & (None, None) \\
    \hline Slice & $(t_{lo}, t_{hi})$ & \hyphen{} & $(t_{lo}, t_{hi})$ \\
    \hline Concat\tnote{1} & $(t_{lo}, t_{hi})\dots$ & \hyphen{} & $(\max(t_{lo}), \max(t_{hi}))$ \\
    \hline Resettable & $(t_{lo}, t_{hi})$ & \hyphen{} & $(t_{lo}, t_{hi})$ \\
    \hline
  \end{tabular}
  \begin{tablenotes}
    \item [1] Note that Concat can operate on any number of inputs.
  \end{tablenotes}
  \end{threeparttable}
  \caption{Cascading of timings}\label{fwd.timings}
\end{table}


\subsection{Input Retimer}\label{algo.timing}
In some cases, the time-setting algorithm does not converge and not all nodes are converted from the \lstinline|flow2| dialect to the \lstinline|flow_timed| dialect. This is as a result of timing constraints not being met, for example nodes may depend on future values or, in the case of an uncompilable input, the value of a node may depend on itself at the current time. Any node which does not depend, including indirect dependence, on its own value at the current time or a future time can be retimed such that the `current' time moves to the time the latest node it depends on becomes ready. This changes the other input nodes to require buffering, and therefore also be ready at the time the calculation is performed.

\subsubsection{Detecting Problematic Offsets}
Problematic offsets appear when times have been assigned such that a node being offset is given a ready time after the ready time of the offset. This can occur as the  This is not possible as it would require an inverse buffer, which doesn't exist. An example of this can be seen in listing \ref{mvg_avg.forward}, with a rewritten version in listing \ref{mvg_avg.backward} \hyphen{} the two listings are equivalent in hardware, however there are implications for timing and buffering. Recalling that the latency between the input and output is not strictly defined, it can be deduced that these graphs produce the same output, however there is still the issue that in the version with the forward offset, there is a situation where the value of \lstinline|t_0| depend on an input value which has not entered the system yet. These issues can be resolved by firstly redefining the origin time $t$ as four clock cycles earlier than it was previously. By adding buffers on the input, it is then possible to get the value of \lstinline|in0| at both $t=0$ and $t=3$. These buffers are added during this step, to produce a modified graph. An equivalent code listing for this can be seen in listing \ref{mvg_avg.modified}. As the ready time of \lstinline|in0_before| is fixed to zero, changing this to effectively be four cycles earlier allows the value which comes in at time zero, previously time four, to be ready for the calculation of \lstinline|t_0|. 

\renewcommand\theFancyVerbLine{\arabic{FancyVerbLine}}
\begin{listing}[H]
  \begin{minted}[numbers=left]{python3}
    avg_window = 4
    in0 = gr.istream(32)
    t_0 = in0.offset(avg_window - 1) - in0
    total = gr.forward_ref("total").offset(-1) + t_0
    c1 = gr.const(avg_window)
    div_t = total / c1
    out0 = gr.ostream(div_t, 32)
  \end{minted}
  \caption{Flow DSL Moving Average: Forward Offset}\label{mvg_avg.forward}
\end{listing}

\renewcommand\theFancyVerbLine{\arabic{FancyVerbLine}}
\begin{listing}[H]
  \begin{minted}[numbers=left]{python3}
    avg_window = 4
    in0 = gr.istream(32)
    t_0 = in0 - in0.offset(-avg_window + 1)
    total = t_0 + gr.forward_ref("total").offset(-1)
    c1 = gr.const(avg_window)
    div_t = total / c1
    out0 = gr.ostream(div_t, 32)
  \end{minted}
  \caption{Flow DSL Moving Average: Backward Offset}\label{mvg_avg.backward}
\end{listing}

\renewcommand\theFancyVerbLine{\arabic{FancyVerbLine}}
\begin{listing}[H]
  \begin{minted}[numbers=left]{python3}
    avg_window = 4
    in0_before = gr.istream(32)
    in0 = in0_before.offset(-avg_window)
    t_0 = in0.offset(avg_window - 1) - in0
    total = gr.forward_ref("total").offset(-1) + t_0
    c1 = gr.const(avg_window)
    div_t = total / c1
    out0 = gr.ostream(div_t, 32)
  \end{minted}
  \caption{Flow DSL Moving Average: Backward Offset, Modified}\label{mvg_avg.modified}
\end{listing}

\subsubsection{Detiming}
When problematic nodes are detected, the timing on any nodes which are `descendant's of the input node which has had an offset inserted after it will have incorrect timing. As there was an offset which violated timing constraints, fixing this is not as simple as just subtracting the time of the offset from each descendant \hyphen{} their timing must be fully re-analysed. To enable this, every node in the graph is replaced with their equivalent from the \lstinline|flow2| dialect. The timing algorithm as described in section \ref{algo.timing} is applied again. While it would be possible to only replace affected nodes, this would mean a new retiming rewrite pattern would need to be implemented, as the predetermined timings for \lstinline|flow_timed| nodes are not considered using the current algorithm.

\subsection{xDSL To Hardware}
To get from a still dataflow-adjacent dialect to a dialect closer to hardware, a new dialect, \lstinline|hard_flow|, is introduced. This dialect replaces the concept of a `Node' with the concept of a `Register' \hyphen{} while they are functionally identical, the dialect departs from the ideas of dataflow programming, therefore the naming convention matches this departure. One feature introduced by the \lstinline|hard_flow| dialect compared to the \lstinline|flow_timed| dialect are buffers, which work atop both offsets and places where implicit buffering is required, such as the example in figure \ref{dataflow.implicit_buffer}. The other is widths for nodes beyond inputs and outputs, so that registers and wires of the correct width can be declared in Verilog. An example of the \lstinline|hard_flow| dialect can be seen in listing \ref{dialect.hard}. As the inner product does not introduce any buffering, examples of a moving average in \lstinline|flow_timed| and \lstinline|hard_flow| are given in listings \ref{dialect.timed.mvg} and \ref{dialect.hard.mvg} respectively. To ensure that timing information is maintained, offsets are maintained. While the offsets do not map to anything in hardware, keeping offsets allows for any discrepancies in argument timing to be detected by the `Flow Printer', rather than assuming the discrepancies are due to offsets.

\renewcommand\theFancyVerbLine{\arabic{FancyVerbLine}}
\makeatletter
\AtBeginEnvironment{minted}{\dontdofcolorbox}
\def\dontdofcolorbox{\renewcommand\fcolorbox[4][]{##4}}
\makeatother
\begin{listing}[H]
  \begin{minted}[numbers=left, breaklines]{llvm}
%0 = "hard_flow.istream"() {"uid" = "in0"} : () -> !hard_flow.reg<#int<0>, #int<32>>
%1 = "hard_flow.istream"() {"uid" = "in1"} : () -> !hard_flow.reg<#int<0>, #int<32>>
%2 = "hard_flow.binop"(%0, %1) {"op" = "*", "uid" = "current"} : (!hard_flow.reg<#int<0>, #int<32>>, !hard_flow.reg<#int<0>, #int<32>>) -> !hard_flow.reg<#int<1>, #int<32>>
%3 = "hard_flow.binop"(%2, %<UNKNOWN>) {"op" = "+", "uid" = "sum_out"} : (!hard_flow.reg<#int<1>, #int<32>>, !hard_flow.reg<#int<1>, #int<32>>) -> !hard_flow.reg<#int<2>, #int<32>>
%4 = "hard_flow.offset"(%3) {"offset" = #int<-1>} : (!hard_flow.reg<#int<2>, #int<32>>) -> !hard_flow.reg<#int<1>, #int<32>>
%5 = "hard_flow.ostream"(%3) {"uid" = "out0"} : (!hard_flow.reg<#int<2>, #int<32>>) -> !hard_flow.reg<#int<2>, #int<32>>
  \end{minted}
  \cprotect\caption{An implementation of an inner product in the \lstinline|hard_flow| dialect}\label{dialect.hard}
\end{listing}

\renewcommand\theFancyVerbLine{\arabic{FancyVerbLine}}
\makeatletter
\AtBeginEnvironment{minted}{\dontdofcolorbox}
\def\dontdofcolorbox{\renewcommand\fcolorbox[4][]{##4}}
\makeatother
\begin{listing}[H]
  \begin{minted}[numbers=left, breaklines]{llvm}
%0 = "flow_timed.istream"() {"uid" = "in0"} : () -> !flow_timed.wnode<#int<0>, #int<32>>
%1 = "flow_timed.offset"(%0) {"offset" = #int<-3>} : (!flow_timed.wnode<#int<0>, #int<32>>) -> !flow_timed.node<#int<-3>>
%2 = "flow_timed.binop"(%0, %1) {"op" = "-", "uid" = "t_0"} : (!flow_timed.wnode<#int<0>, #int<32>>, !flow_timed.node<#int<-3>>) -> !flow_timed.node<#int<1>>
%3 = "flow_timed.binop"(%2, %4>) {"op" = "+", "uid" = "total"} : (!flow_timed.node<#int<1>>, !flow_timed.node<#int<1>>) -> !flow_timed.node<#int<2>>
%4 = "flow_timed.offset"(%3) {"offset" = #int<-1>} : (!flow_timed.node<#int<2>>) -> !flow_timed.node<#int<1>>
%5 = "flow_timed.const"() {"value" = #int<4>, "uid" = "c1"} : () -> !flow_timed.const_t<#int<32>>
%6 = "flow_timed.binop"(%3, %5) {"op" = "/", "uid" = "div_t"} : (!flow_timed.node<#int<2>>, !flow_timed.const_t<#int<32>>) -> !flow_timed.node<#int<3>>
%7 = "flow_timed.ostream"(%6) {"uid" = "out0"} : (!flow_timed.node<#int<3>>) -> !flow_timed.wnode<#int<3>, #int<32>>
  \end{minted}
  \cprotect\caption{An implementation of a moving average in the \lstinline|flow_timed| dialect}\label{dialect.timed.mvg}
\end{listing}

\renewcommand\theFancyVerbLine{\arabic{FancyVerbLine}}
\makeatletter
\AtBeginEnvironment{minted}{\dontdofcolorbox}
\def\dontdofcolorbox{\renewcommand\fcolorbox[4][]{##4}}
\makeatother
\begin{listing}[H]
  \begin{minted}[numbers=left, breaklines]{llvm}
%0 = "hard_flow.istream"() {"uid" = "in0"} : () -> !hard_flow.reg<#int<0>, #int<32>>
%1 = "hard_flow.offset"(%0) {"offset" = #int<-3>} : (!hard_flow.reg<#int<0>, #int<32>>) -> !hard_flow.reg<#int<-3>, #int<35>>
%2 = "hard_flow.buffer"(%1) {"by" = #int<3>} : (!hard_flow.reg<#int<-3>, #int<35>>) -> !hard_flow.reg<#int<0>, #int<35>>
%3 = "hard_flow.binop"(%0, %2) {"op" = "-", "uid" = "t_0"} : (!hard_flow.reg<#int<0>, #int<32>>, !hard_flow.reg<#int<0>, #int<35>>) -> !hard_flow.reg<#int<1>, #int<35>>
%4 = "hard_flow.binop"(%3, %5) {"op" = "+", "uid" = "total"} : (!hard_flow.reg<#int<1>, #int<35>>, !hard_flow.reg<#int<1>, #int<35>>) -> !hard_flow.reg<#int<2>, #int<35>>
%5 = "hard_flow.offset"(%4) {"offset" = #int<-1>} : (!hard_flow.reg<#int<2>, #int<35>>) -> !hard_flow.reg<#int<1>, #int<35>>
%6 = "hard_flow.const"() {"value" = #int<4>, "uid" = "c1"} : () -> !hard_flow.const_t<#int<3>>
%7 = "hard_flow.binop"(%4, %6) {"op" = "/", "uid" = "div_t"} : (!hard_flow.reg<#int<2>, #int<35>>, !hard_flow.const_t<#int<3>>) -> !hard_flow.reg<#int<3>, #int<32>>
%8 = "hard_flow.ostream"(%7) {"uid" = "out0"} : (!hard_flow.reg<#int<3>, #int<32>>) -> !hard_flow.reg<#int<3>, #int<32>>
  \end{minted}
  \cprotect\caption{An implementation of a moving average in the \lstinline|hard_flow| dialect}\label{dialect.hard.mvg}
\end{listing}

\begin{figure}[h]
  \centering
  \begin{subfigure}[t]{0.5\textwidth}\centering
    \begin{tikzpicture}[node distance=1cm,circuit ee IEC]
      % below n
      \node (x) [xshift=0cm]{$x_0$};
      \node (y) [right=of x,xshift=-0.2cm]{$y_0$};
      \node (add_0) [op,below=of y,label={[label distance=-0.15cm]south east: \scriptsize 1}]{$+$};
      \node (add_1) [op, below left=of add_0,label={[label distance=-0.15cm]south east: \scriptsize 2}]{$+$};
      \node (out) [below=of add_1] {$out_2$};
      \draw[->](x)--(add_0);
      \draw[->](y)--(add_0);
      \draw[->](x)--(add_1);
      \draw[->](add_0)--(add_1);
      \draw[->](add_1)--(out);
    \end{tikzpicture}
    \caption{Without implicit buffering}
  \end{subfigure}%
  \begin{subfigure}[t]{0.5\textwidth}\centering
    \begin{tikzpicture}[node distance=1cm,circuit ee IEC]
      \node (x) [xshift=0cm]{$x_0$};
      \node (y) [right=of x,xshift=-0.2cm]{$y_0$};
      \node (add_0) [op,below=of y,label={[label distance=-0.15cm]south east: \scriptsize 1}]{$+$};
      \node (buf_0) [op,below=of x,label={[label distance=-0.15cm]south east: \scriptsize 1}]{nop};
      \node (add_1) [op, below=of buf_0,label={[label distance=-0.15cm]south east: \scriptsize 2}]{$+$};
      \node (out) [below=of add_1] {$out_2$};
      \draw[->](x)--(add_0);
      \draw[->](x)--(buf_0);
      \draw[->](y)--(add_0);
      \draw[->](add_0)--(add_1);
      \draw[->](buf_0)--(add_1);
      \draw[->](add_1)--(out);
    \end{tikzpicture}
    \caption{With implicit buffering}
  \end{subfigure}
  \caption{Offset replacement: timed dataflow graphs} \label{dataflow.implicit_buffer}
\end{figure}

\subsubsection*{Introduction of Buffers}
The introduction of explicit buffers marks the largest departure from the IR code from the dataflow programming model. Buffers are used to store values between when they are created and when they are used. It is important to note that an offset doesn't imply a buffer, and the lack of an offset doesn't imply the lack of a buffer. As shown in figure \ref{dataflow.implicit_buffer}, buffers can be implicit as a result of an operation being performed with inputs ready at different times, and as in the Flow DSL. As shown in figure \ref{dataflow.no_buffer}, offsets which result in a timing slack of zero \hyphen{} that is the offset results in the input to an operation being ready one cycle before the output, an offset can result in no buffering. Buffers are introduced with the formula $b_{arg} = t_{op} - t_{arg}$, where $op$ has $arg$ as an argument and $b_{arg}$ is the number of buffers required for $arg$. As $arg$ may be an argument for multiple operations, the final value $b_{arg}$ is determined by the maximum value for any operation which $arg$ is an argument to. As constants do not change per-cycle, they are immune from buffering and so $b_{const} = 0$ $\forall$ $const$.

\begin{figure}[h]
  \centering
  \begin{subfigure}[t]{0.5\textwidth}\centering
    \caption{Without implicit buffering}
    \begin{tikzpicture}[node distance=1cm,circuit ee IEC]
      % below n
      \node (x) [xshift=0cm]{$x_0$};
      \node (add_0) [op,below=of x,label={[label distance=-0.15cm]south east: \scriptsize 1}]{$+$};
      \node (buf_0) [buf, right=of add_0,label={[label distance=-0.15cm]south east: \scriptsize 0}]{last};
      \node (out) [below=of add_0] {$out_1$};
      \draw[->](x)--(add_0);
      \path
        (add_0)[->,bend right] edge node [right] {} (buf_0)
        (buf_0)[->,bend right] edge node [left] {} (add_0);
      \draw[->](add_0)--(out);
    \end{tikzpicture}
  \end{subfigure}%
  \begin{subfigure}[t]{0.5\textwidth}\centering
    \caption{With implicit buffering}
    \begin{tikzpicture}[node distance=1cm,circuit ee IEC]
      % below n
      \node (x) [xshift=0cm]{$x_0$};
      \node (add_0) [op,below=of x,label={[label distance=-0.15cm]south east: \scriptsize 1}]{$+$};
      \node (out) [below=of add_0] {$out_1$};
      \draw[->](x)--(add_0);
      \path (add_0)[->, loop right,in=25,out=-25,min distance=0.3cm,looseness=10] edge [right] node [right] {} (add_0);
      \draw[->](add_0)--(out);
    \end{tikzpicture}
  \end{subfigure}
  \caption{Offset replacement: timed dataflow graphs}\label{dataflow.no_buffer}
\end{figure}

\subsubsection{Setting Widths}
Initially, widths are only set for inputs and outputs \hyphen{} either explicitly by the user or by using the default of four bytes. four bytes was chosen as this is the default size of an integer in most programming languages, however a single byte also would've been a valid choice as this is frequently used in image processing, parallel communication and other domains.

For RTL code to be generated, widths are needed for every node of the graph. While the default sizes for binary operations are written in \ref{embedding.binops.width}, this oversimplifies the concept. Considering the operation \lstinline|x = gr.forward_ref("x").offset(-2) * y|, and applying the formula for multiplication, $w_x = w_x + w_y$, we find that $w_x$ will tend to infinity as part of an iterative process. We therefore not only need to generate widths based on the inputs available to them, but also the outputs used. These widths based on output usage can be seen in table \ref{widths.output}. Note that the  While additional challenges are provided by logical operators such as \lstinline||||, \lstinline|&&|, \lstinline|!| amongst others, for most cases by combining the bits available to an operation with the bits used by an operation, we will reach a single value for the width of the node's output \hyphen{} this will come from the maximum value in the set of used bits.

\begin{table}[h]
  \centering
  \begin{threeparttable}
  \begin{tabular}{|c|c|c|c|}
    \hline
    \textbf{Operator} & \textbf{Available Bits} & \textbf{Used Input Bits} & \textbf{Used Bits} \\
    \hline \lstinline|&| (BAND) & $w_1$ & $w_2$ & $w_2$ \\
    \hline \lstinline||| (BOR) & $w_1$ & $w_2$ & $w_2$ \\
    \hline \lstinline|^| (XOR) & $w_1$ & $w_2$ & $w_2$ \\
    \hline \lstinline|+| (ADD) & $w_1$ & $w_2$ & $w_2$ \\
    \hline \lstinline|-| (SUB) & $w_1$ & $w_2$ & $w_2$ \\
    \hline \lstinline|*| (MUL) & $w_1$ & $w_2$ & $w_2$ \\
    \hline \lstinline|/| (DIV) & $w_1$ & $w_2$ & $w_1$ $\cup$ $w_2$ \\
    \hline \lstinline|%| (MOD) & $w_1$ & $w_2$ & $w_1$ $\cup$ $w_2$ \\
    \hline \lstinline|<<| (LSL)\tnote{1} & $w_1$ & $w_2$ & $w$ $\forall$ $w \ge c$ $\in$ $w_2$  \\
    \hline \lstinline|>>| (LSR)\tnote{1} & $w_1$ & $w_2$ & $w + c$ $\forall$ $w$ $\in$ $w_2$  \\
    \hline \lstinline|and| (LAND) & $w_1$ & $w_2$ & $w_1$ \\
    \hline \lstinline|or| (LOR) & $w_1$ & $w_2$ & $w_1$ \\
    \hline \lstinline|==| (EQ) & $w_1$ & $w_2$ & $w_1$ \\
    \hline \lstinline|!=| (NEQ) & $w_1$ & $w_2$ & $w_1$ \\
    \hline \lstinline|>=| (GE) & $w_1$ & $w_2$ & $w_1$ \\
    \hline \lstinline|>| (GT) & $w_1$ & $w_2$ & $w_1$ \\
    \hline \lstinline|<=| (LE) & $w_1$ & $w_2$ & $w_1$ \\
    \hline \lstinline|<| (LT) & $w_1$ & $w_2$ & $w_1$ \\
    \hline \lstinline|@| (CAT)\tnote{2} & $w_1$ & $w_2$ & $w_1$ \\
    \hline \lstinline|[c1:c2]| (SLICE) & $w_1$ & $c_1, c_2$ & $x \forall c_1 \le x < c_2 \in \mathbb{Z}$ \\
    \hline
  \end{tabular}
  \begin{tablenotes}
    \item [1] $c$ is the shift amount, which must be constant
    \item [2] Cat is a special case, where more bits than used may be output to allow consistent indexing
  \end{tablenotes}
  \end{threeparttable}
  \caption{Default bit width of binary operators based on output widths}\label{widths.output}
\end{table}

\subsection{Flow Printer}\label{sect:compiling.flow_printer}
While it would have been ideal to extend xDSL's \lstinline|Printer| class, which already contains mechanisms for keeping track of the names of Single Static Assignment Values (SSAValues) among other functionality useful for outputting the target language, in this case Verilog, this was not possible due to Python's poor support for subclassing and inheritance \hyphen{} there were some key restrictions in how methods were called which stopped the extended version from working. It was therefore necessary to implement a new printer, \lstinline|FlowPrinter|, to take \lstinline|hard_flow| code and output Verilog.

While there are no optimisations as such applied to the \lstinline|hard_flow| code, the flow printer still contains a number of transformation algorithms necessary to bridge the gap between the IR and Verilog.

\subsubsection{Naming Variables}\label{naming}
As variables may have been merged, significantly reordered, erased or otherwise modified in such a way that their name has not been retained, and variables which had generated names assigned in section \ref{graph.2.xdsl} may have lost their names, new names are assigned to any non-buffer variables without a \lstinline|uid| return type attribute. The `print context' keeps track of the names of both named and unnamed variables, and if a the name for a new unnamed variable is requested, a name is assigned in the form \lstinline|gen_{i}|, where $i$ is a sequentially increasing integer.

As forward references may still be present, the list of nodes is first iterated though and an initial call to \lstinline|PrintContext.get_name(SSAValue)| is applied to each \lstinline|SSAValue|, aka node return value, in the graph. This introduces each variable into the print context and assigns them a name.

One exception to variables being given a name are constants. Constants are invoked directly as their integral value with their correct width, rather than connecting them as registers or wires which may serve to complicate the design, especially if a single constant assigned to a single net is used in multiple operations \hyphen{} this may lead to fairly complex interconnects. Another exception to variables being named in the above form are buffers. Buffers are named in the form \lstinline|buffer_{node}_{t}|, where $node$ is the name of the node being buffered, and $t$ is the number of clock cycles after the initial calculation \hyphen{} this may include generated nodes and offsets, but not constants.

\subsubsection{Applying Buffers}
Buffers are explicitly included in \lstinline|hard_flow|, therefore while they do not need to be generated here, they follow a different naming system to other SSAValues, as detailed in \ref{naming}. The buffers are kept track of in the `Print Context' aside from other variables, so that they can be generated, propagated, initialised and added to the output without needing to recalculate them each time.

\subsubsection{Implicit Reset}\label{reset.implicit}
To honour the commitment to one input per clock cycle while still allowing accumulators and other user-selected signals to be affected by a reset signal, an implicit reset is generated. With the timing algorithm in its current form, it is only possible to add buffers, not to remove them, therefore to allow for accumulators and other `reduce' or `scan' style functions to work, a special zero clock cycle operation needs to be introduced, as the register \lstinline|acc| in \lstinline|acc = (gr.forward_ref('acc').offset(-1) + in0) & reset_n| would be a timing violation, as the result would be ready one clock cycle later than it needs to be.

The reset signal cannot reset the registers on the next clock cycle, as that would require them to be held low for an entire clock cycle, causing those inputs to be missed. To honour the reset signal and the one input per clock concept at the same time, we can buffer the reset signal and apply it on inputs to nodes, rather than as inputs to registers. An example of this can be seen in listing \ref{compiled.inner_product}, where a simple add based accumulator is used with a reset to 0 on the input to the node \lstinline|sum_out|.

\renewcommand\theFancyVerbLine{\arabic{FancyVerbLine}}
\begin{listing}[H]
  \begin{minted}[numbers=left]{verilog}
module GeneratedModule(
    input clk,
    input reset_n,
    input [31:0] in0,
    input [31:0] in1,
    output [31:0] out0
);
  reg buffer_reset_n_1;
  reg [31:0] current;
  reg [31:0] sum_out;
  wire [31:0] gen_0 = buffer_reset_n_1? sum_out : 0;
  initial begin
    buffer_reset_n_1 = 0;
    current = 0;
    sum_out = 0;
  end
  assign out0 = sum_out;
  always @(posedge clk) begin
    buffer_reset_n_1 <= reset_n;
    current <= in0 * in1;
    sum_out <= gen_0 + current;
  end
endmodule
\end{minted}
  \caption{Inner product compiled from Flow DSL to Verilog}\label{compiled.inner_product}
\end{listing}

\subsubsection{Output}
The output language is Verilog, and so the output follows the syntax of Verilog. This means that first the inputs and outputs, plus an implicit \lstinline|clock| and \lstinline|reset_n| are printed \hyphen{} these are all `wires', however the output is most frequently connected directly to a register, effectively functioning as an `output reg'. Once the module header is printed, buffers for the reset signal are added \hyphen{} these are based on when the last \lstinline|resettable| or accumulator node is located, and if there are no \lstinline|resettable| nodes this step will be skipped. All buffering nodes are also added here, with a data structure to keep track of their initialisation values for simulation, otherwise the values of cyclic graphs would not propagate in Verilog. The binary and unary operation results are also declared as registers here.

After all the registers have been declared, an \lstinline|initial| block is used to initialise all buffers to zero. This is the only time that a value not specified in the dataflow graph is written to a buffer \hyphen{} when \lstinline|reset_n| is high, the value is the buffer is updated as normal, with the update value potentially changing from the normal operation of the graph if there is a \lstinline|resettable| input.

Following the register declaration and initialisation, wires are declared. These wires are only used for \lstinline|resettable| inputs \hyphen{} constants and offsets are written inline.

The final part of the Verilog program is to update the buffers and outputs from binary operations. A clocked block, declared with \lstinline|always @(posedge clk)|, is used to enable this. Each buffer is cascaded with the next value being the current value of the previous buffer, or the input to the buffer if it is the first value in the chain, in the form \lstinline|buffer_(nd)_(i) <= buffer_(nd)_(i-1)|. After buffer updates are output, binary operations are output, with the register being updated with the result of the binary operation.

Currently there is no support for writing directly to a file as this was deemed unnecessary for testing purposes, however this could easily be added by giving the `Flow Printer' a output stream \hyphen{} this could be standard output, should the user desire the resulting Verilog to be written to the console.

\par\noindent\hrulefill\par

In this section, the full compilation process of Flow DSL was detailed, including some details of where the compilation process would ideally have been done differently. The method for outputting the resultant Verilog from the IR was detailed.

% Review
\chapter{Evaluation}
In this chapter, Flow DSL is compared to both the requirements set in section \ref{chap:requirements} and the existing HDLs Chisel and Verilog, then the results of these comparisons analysed and considered.

\section{Research Question}
Static streaming dataflow optimisations have been found to have been a useful tool for hardware compilation, enabling a number of optimisations to be applied. This will have reaching consequences and potential applications in a number of hardware compilers, even those which do not explicitly use dataflow, due to the simplicity of the system allowing for easy inclusion in existing compilation processes.

\section{Comparison to Requirements}
\subsection{Principles of Language Design}
The Gaussian blur code in Flow DSL is relatively simple and intuitive. It leverages Python's syntax, making it easy for users familiar with Python to understand. The code defines dataflow graphs in a minimalistic way, which aligns with the simplicity principle of Flow DSL.

The code is expressive enough to capture a complex dataflow graph for Gaussian blur, as in listing \ref{python.gb.kernel}. It currently only supports unsigned integers as data types, therefore the requirement to support various data types is not met, however it does include a number of arithmetic operations as well as accumulators, as well as more user-defined cycle forms.

\subsection{High Level Requirements}
Inputs and outputs are both easily declared via the \lstinline|Graph| object, which is in line with the high level requirements. The DSL also allows for declaration of a dataflow graph in a high-level language without specifying the timing, again meeting the requirements. It focuses on the logical sequence of data transformation without being concerned about the temporal aspects.

Resets \hyphen{} or rather holding signals to a set value, are possible within a single clock cycle due to the mechanisms described in section \ref{reset.implicit}. While this started out of a stopgap solution to allow for user-defined accumulator functions in a `reduce' or `fold' format, this is a useful feature to have, as it allows for the dataflow graph to be reset to the initial `state' without needing to add clocked logic. This does stray slightly from the concept of `static' streaming dataflow graphs, however the ability to batch dataflow computations is useful eg. when processing multiple images.

The Gaussian blur is a type of stencil computation, and the code in listing \ref{python.gb.kernel} demonstrates how Flow DSL can be used for such computations and more. It involves summation and multiplication, as necessitated by the requirements.

\subsection{Dataflow}
Code written in the DSL specifies the structure of the dataflow graph (inputs, outputs, and operations) but does not specify timing, allowing for potential optimizations such as retiming and buffering by the compiler.

The compiler itself is able to apply some retiming, adding buffers as necessary, however it currently doesn't possess the ability to remove buffers except for in very specific situations \hyphen{} currently only when a constant value is involved, or when an offset is used such that the value needs no buffering between the time it becomes available and the time it is used. The implications of this and potential fixes are discussed further in section \ref{ext.retime}.

\subsection{Streaming}
The DSL compiler assumes that inputs are populated and outputs are extracted at regular intervals \hyphen{} once per clock cycle. However, the DSL itself does not explicitly handle clock dividers or counters. The document mentions that the DSL should assume a fixed one value per input stream per clock cycle, and the code is written with this assumption.

\section{Demonstration and Comparison to Other HDLs}
\subsection{Roberts Cross Edge Detector}
The Roberts Cross edge detector is an edge detector which is particularly well-suited for edge detection in a live image stream with a low degree of noise due to its simplicity and low hardware usage. listing \ref{python.roberts_cross} shows an implementation in Flow DSL, and listing \ref{verilog.roberts_cross} shows the Verilog alternative. It uses two kernels of size 2x2, and the difference is calculated to detect the edges.

\begin{listing}[H]
  \begin{minted}[numbers=left]{python3}
screen_height = 480

r_in = gr.istream(8)
g_in = gr.istream(8)
b_in = gr.istream(8)

def roberts_cross(in0: Node) -> Node:
  last = in0.offset(-1)
  prev_row = in0.offset(-screen_height)
  prev_row_last = last.offset(-screen_height)
  gx = prev_row_last - in0
  gy = prev_row - last
  return gx + gy

r_out = gr.ostream(roberts_cross(r_in), 8)
g_out = gr.ostream(roberts_cross(g_in), 8)
b_out = gr.ostream(roberts_cross(b_in), 8)
  \end{minted}
  \caption{Flow DSL implementation of a Roberts Cross edge detector}\label{python.roberts_cross}
\end{listing}

\begin{listing}[H]
  \begin{minted}[numbers=left, breaklines]{verilog}
module RobertsCrossEdgeDetector (
  input [7:0] r_in, g_in, b_in,
  input clk, reset_n,
  output reg [7:0] r_out, g_out, b_out
);
  reg [7:0] r_line_buffer_0[479:0], r_line_buffer_1[479:0];
  reg [7:0] g_line_buffer_0[479:0], g_line_buffer_1[479:0];
  reg [7:0] b_line_buffer_0[479:0], b_line_buffer_1[479:0];
  wire [7:0] r_px00, r_px01, r_px10, r_px11;
  wire [7:0] g_px00, g_px01, g_px10, g_px11;
  wire [7:0] b_px00, b_px01, b_px10, b_px11;
  reg [8:0] pos = 0;
  assign r_px00 = r_line_buffer_0[pos[8:0]];
  assign r_px01 = r_line_buffer_0[pos[8:0] + 1];
  assign r_px10 = r_line_buffer_1[pos[8:0]];
  assign r_px11 = r_line_buffer_1[pos[8:0] + 1];
  assign g_px00 = g_line_buffer_0[pos[8:0]];
  assign g_px01 = g_line_buffer_0[pos[8:0] + 1];
  assign g_px10 = g_line_buffer_1[pos[8:0]];
  assign g_px11 = g_line_buffer_1[pos[8:0] + 1];
  assign b_px00 = b_line_buffer_0[pos[8:0]];
  assign b_px01 = b_line_buffer_0[pos[8:0] + 1];
  assign b_px10 = b_line_buffer_1[pos[8:0]];
  assign b_px11 = b_line_buffer_1[pos[8:0] + 1];

  integer i;
  always @(posedge clk) begin
    r_line_buffer_1[pos[8:0]] <= r_in;
    g_line_buffer_1[pos[8:0]] <= g_in;
    b_line_buffer_1[pos[8:0]] <= b_in;
    if (pos[8:0] == 479) begin
      for (i = 0; i < 480; i = i+1) begin
          r_line_buffer_0[i] <= r_line_buffer_1[i];
          g_line_buffer_0[i] <= g_line_buffer_1[i];
          b_line_buffer_0[i] <= b_line_buffer_1[i];
      end
      pos <= 0;
    end
    else begin
      r_out <= ((r_px00 >= r_px11) ? (r_px00 - r_px11) : (r_px11 - r_px00)) + ((r_px10 >= r_px01) ? (r_px10 - r_px01) : (r_px01 - r_px10));
      g_out <= ((g_px00 >= g_px11) ? (g_px00 - g_px11) : (g_px11 - g_px00)) + ((g_px10 >= g_px01) ? (g_px10 - g_px01) : (g_px01 - g_px10));
      b_out <= ((b_px00 >= b_px11) ? (b_px00 - b_px11) : (b_px11 - b_px00)) + ((b_px10 >= b_px01) ? (b_px10 - b_px01) : (b_px01 - b_px10));
      pos <= pos + 1;
    end
  end
endmodule
  \end{minted}
  \caption{Verilog implementation of a Roberts Cross edge detector}\label{verilog.roberts_cross}
\end{listing}

When compiled to logic using using the `Analysis \& Elaboration' feature of Intel's `Quartus Prime'\cite{quartus}, but before the `place and route' compilation stage to avoid wiring between assigned IO pins and other FPGA-specific limitations, the design produced using the Flow DSL used 236 logic elements, 225 registers and 11,424 memory bits. In identical conditions, the Verilog design used 40,472 logic elements, 23,107 registers and 12,288 memory bits. While this may seem extreme at first, rather than being an accurate measure of how much more efficient in terms of resource usge Flow DSL is than Verilog, it exposes inefficiencies which are easily missed in the Verilog. Consider the if/else statements on line 31 of listing \ref{verilog.roberts_cross} \hyphen{} each statement within this will produce a MUX in the form \lstinline|line_buffer_0[i] <= pos[8:0] == 479?  line_buffer_1[i] : line_buffer_0[i];|. As there are 480 pixels per line and three colours per pixels, this alone will use 1,440 logic elements \hyphen{} already more than Flow DSL. The vast majority of the logic element usage, however, comes from indexing the row buffer directly, as in the line \lstinline|line_buffer_0[pos[8:0] + 1]|. This line will initially create a MUX for each of the twelve pixels to consider. Each of these MUXes will use around 3000 logic elements, resulting in a highly inefficient end product.

While the throughput of the designs were identical \hyphen{} the requirement of one pixel per clock ensures this, the latency of the design produced by Flow DSL was three clock cycles as opposed to the two from the plain Verilog. This is offset by the maximum clock speed ($F_{max}$) when using a DE10-Lite development kit \cite{terasic} being 69.71MHz for the Verilog design and 347.95MHz for the Flow DSL design due to the setup and hold time limitations for the Verilog design being increased by all the logic which slows propagation through the unneeded MUXes.

Considering improved Verilog code, such as that in listing \ref{verilog.roberts_cross.improved}, which uses modules for reusability and the same buffering system as the code generated by Flow DSL \hyphen{} even if this is less immediately obvious as it is decoupled from the concept of a line in an image, it is similarly efficient in terms of resource usage. The new Verilog code uses 286 logic elements, 21.2\% more than the code generated by FlowDSL, this is offset by handwritten Verilog fewer registers at 81 compared to Flow DSL's 225, meaning Flow DSL used 178\% more registers than handwritten Verilog. The number of memory bits is identical at 11,424. The $F_{max}$ for the DE10-Lite is again lower than that generated by Flow DSL at 204.79MHz \hyphen{} this is only 58.9\% of the clock rate of the design generated by Flow DSL. This shows that Flow DSL's focus on long pipelines has a positive impact when compared to Verilog code which does not focus on long pipelines.

\begin{listing}[H]
  \begin{minted}[numbers=left, breaklines]{verilog}
module RobertsCross(
   input [7:0] in,
   input clk, reset_n,
   output reg [7:0] out
);
  reg [7:0] line_buffer[480:0];
  wire [7:0] px00 = line_buffer[480],
             px01 = line_buffer[479],
             px10 = line_buffer[0],
             px11 = in;
  integer j;
  always @(posedge clk) begin
    line_buffer[0] <= in;
    for (j = 0; j < 479; j = j+1) begin
      line_buffer[j+1] <= line_buffer[j];
    end
   out <= ((px00 >= px11) ? (px00 - px11) : (px11 - px00)) + ((px10 >= px01) ? (px10 - px01) : (px01 - px10));
  end
endmodule

module GeneratedModule (
    input [7:0] r_in, g_in, b_in,
    input clk, reset_n,
    output [7:0] r_out, g_out, b_out
);
  RobertsCross red(
  .in(r_in),
  .clk(clk),
  .reset_n(reset_n),
  .out(r_out)
  );
  RobertsCross green(
  .in(g_in),
  .clk(clk),
  .reset_n(reset_n),
  .out(g_out)
  );
  RobertsCross blue(
  .in(b_in),
  .clk(clk),
  .reset_n(reset_n),
  .out(b_out)
  );
endmodule
\end{minted}
  \caption{Improved Verilog implementation of a Roberts Cross edge detector}\label{verilog.roberts_cross.improved}
\end{listing}

Writing the equivalent code to Flow DSL in Chisel, with the same timing \hyphen{} 3 clock cycles of latency, as in listing \ref{chisel.robertscross} and generating Verilog would be expected to produce similar code to handcrafting Verilog, as Chisel does not optimise RTL, however the code generated by Chisel was actually more efficient than Verilog, using 232 logic elements to Verilog's 286 and 201 registers to Verilog's 81. These figures are both also lower than Flow DSL's 236 and 225 respectively. The $F_{max}$ of the Chisel generated code was 315.06MHz \hyphen{} slower than Flow DSL's 347.95 but still must faster than Verilog. The reduced number of registers is in part due to mis-optimisation in Flow DSL \hyphen{} it performs an optimisation to reduce buffering in the calculation of \lstinline|gx| and so uses the input directly for both \lstinline|gx| and \lstinline|gy|, adding a buffer after \lstinline|gx| and removing one before. As the result of offsetting \lstinline|in0| by 481 is still used by \lstinline|gy|, this `optimisation' to the calculation of \lstinline|gx| actually adds a buffer compared to not applying it, as it becomes necessary to store the result of \lstinline|gy| for an extra clock cycle. To avoid this, it would be beneficial to add a case to catch this in Flow DSL.

The simplicity of Chisel is also closer to that of Flow DSL \hyphen{} as it has the \lstinline|RegNext| type it allows for far easier construction of buffers than Verilog. This simplicity makes construction of pipelined designs easier.

\begin{listing}[H]
  \begin{minted}[numbers=left, breaklines]{scala}
class GeneratedModule extends chisel3.Module {
  val io = IO(new Bundle {
    val r_in = Input(UInt(8.W))
    val g_in = Input(UInt(8.W))
    val b_in = Input(UInt(8.W))
    val r_out = Output(UInt(8.W))
    val g_out = Output(UInt(8.W))
    val b_out = Output(UInt(8.W))
  })
  import io._
  final val ScreenWidth = 480
  def genCross(in: UInt): UInt = {
    lazy val delays: LazyList[UInt] =
      in +: LazyList.from(0).map(i => RegNext(delays(i)))
    val gx = RegNext((delays(ScreenWidth + 1) - delays(0)).asSInt)
    val gy = RegNext((delays(1) - delays(ScreenWidth)).asSInt)
    (RegNext{gx.abs} + RegNext(gy.abs)).asUInt
  }
  r_out := RegNext(genCross(r_in))
  g_out := RegNext(genCross(g_in))
  b_out := RegNext(genCross(b_in))
}
\end{minted}
  \caption{Chisel implementation of a Roberts Cross edge detector}\label{chisel.robertscross}
\end{listing}

\subsection{General Purpose Kernels}
Something which is not possible in languages such as Verilog, but is possible in Flow DSL, is the creation of a general purpose image processing kernel applier. This can result is extremely low effort required to modify designs in terms of size, kernel parameters and number of kernels. An implementation is given in listing \ref{python.kernel}. As this is not possible in Verilog, a comparison implementation cannot be given \hyphen{} it would be necessary to write a new implementation almost from scratch for each kernel.

\begin{listing}[H]
  \begin{minted}[numbers=left]{python}
def apply_kernel(
    gr: Graph, in0: Node, kernel: list[list[int]], divisor: int = None
) -> Node:
  assert len(kernel)  > 0
  assert len(kernel[0]) > 0
  mid_y = (len(kernel) + 1) // 2
  mid_x = (len(kernel[0]) + 1) // 2
  divisor = sum(map(sum, kernel))
  out = None
  for (i, x) in enumerate(kernel):
    for (j, y) in enumerate(x):
      if y != 0:
        offset_by = (i - mid_y) * screen_width + (j - mid_x)
        node = in0.offset(offset_by) * gr.const(y)
        if out != None:
          out = out + node
        else:
          out = node
  if out != None:
    if divisor:
      return out / gr.const(divisor)
    else:
      return out
  else:
    return gr.const(0)
  \end{minted}
  \caption{Application of a generic kernel in Flow DSL}\label{python.kernel}
\end{listing}


\subsection{Alteration of Existing Code}
In Flow DSL, alteration of existing code is inherently simple due to the high-level nature of the dataflow graphs and the ability to declare and invoke Python functions. Using the example of the Roberts Cross in listing \ref{python.roberts_cross}, changing the image width merely requires updating the \lstinline|screen_width| variable. The underlying dataflow graph remains the same other than the buffer length, and the compiler handles all buffering changes internally. More major alterations are also simple in Flow DSL; modifying the kernel or switching to a different image processing algorithm, such as the Gaussian Blur operator, as shown listing \ref{python.gb.kernel}, can be done with minimal changes to the code for a Roberts Cross operator, as seen in listing \ref{python.rc.kernel}. The \lstinline|apply_kernel| function demonstrated in listing \ref{python.kernel} is a prime example of adaptability, as it allows for the application of arbitrary convolution kernels with little effort.

On the other hand, in Verilog, alteration of existing code in similar manners often requires more extensive modifications. Due to the lower-level nature of Verilog, handling different image sizes might necessitate changes in buffer sizes, indexing, and control logic. Adaptability in Verilog is also more challenging. Changing the kernel or algorithm usually requires a deeper understanding of the hardware and involves modifying multiple sections of the code, including the arithmetic operations and control structures. This can be error-prone and time-consuming.

As Chisel allows dataflow-like operations in terms of `RegNext', a similar kernel application function would be possible to write in Chisel, however it is worth noting that as Chisel only allows accessing the last value using `RegNext', buffering LazyLists would need to be created to input buffer chains such as though automatically generated by Flow DSL when changing the level of, or introducing, buffering.

\begin{listing}[H]
  \begin{minted}[numbers=left]{python}
def roberts_cross(gr: Graph, in0: Node):
  a = apply_kernel(gr, in0, [[-1, 0], [0, -1]])
  b = apply_kernel(gr, in0, [[0, 1], [1, 0]])
  return abs(a) + abs(b)
  \end{minted}
  \caption{Application of a Roberts Cross using the `apply kernel' function}\label{python.rc.kernel}
\end{listing}

\begin{listing}[H]
  \begin{minted}[numbers=left]{python}
def gaussian_blur(gr: Graph, in0: Node):
  return (
    apply_kernel(gr, in0, [[1, 2, 1],
                           [2, 4, 2],
                           [1, 2, 1]], 16)
  )
  \end{minted}
  \caption{Application of a Gaussian Blur using the `apply kernel' function}\label{python.gb.kernel}
\end{listing}

\par\noindent\hrulefill\par

In this section, Flow DSL was compared in terms of both performance and modifiability to Verilog and Chisel, and against the requirements. It was found to be relatively performant, especially compared to hand-written Verilog, and met most of the requirements.
\chapter{Future Extensions}
This section details additions which, given more time, would have been good to add to Flow DSL. These range from expansions to fixes to further optimisations. While Flow DSL in its current form is a useful tool, the addition of these fixes would make it far better.

\section{Replacement of the retiming algorithm}\label{ext.retime}
The current implementation of the retiming algorithm has several limitations which results in some dataflow graphs with low dependency distance across cycles, such as that in listing \ref{notworking}, not being possible to compile. This is because the current timing algorithm cannot retime operations to take less than a single clock cycle \hyphen{} buffering expected after every arithmetic operation.

The proposed replacement for this timing algorithm is a new system wherein each node is given a current time, a minimum and a maximum time, with the minimum and maximum times being either relative to the time of its inputs becoming ready or absolute. Initially all current times will be zero, and all minimum and maximum times will be \lstinline|None|. In non-cyclic graphs, the maximum time will always be \lstinline|None|, however in cyclic graphs, some nodes can be expected to have a maximum ready time depending on the dependency distance. Each operation will have a desired length \hyphen{} allowing for complex operations like division and multiplication to be pipelined across multiple clock cycles, and the per-node timing tweaked in such a way that the timing is as close to this target length of time after the timing of its inputs as possible without going over the maximum ready time. The minimum ready time will be one after its inputs in the case of accumulators and immediate cycles to prevent values from being self-dependant, or in most cases will be zero.

This replacement would allow for far more powerful optimisations and pipelining to be used, as well as allowing for more dataflow graphs than currently possible to be compiled.

\makeatletter
\AtBeginEnvironment{minted}{\dontdofcolorbox}
\def\dontdofcolorbox{\renewcommand\fcolorbox[4][]{##4}}
\makeatother
\begin{listing}[H]
  \begin{minted}[numbers=left, breaklines]{python}
x = gr.istream()
y = gr.istream()
z = gr.ostream((gr.forward_ref("a").offset(-1) + x) * y)
  \end{minted}
  \cprotect\caption{A sub-clock cycle operation}\label{notworking}
\end{listing}

\section{More types}
Flow DSL currently only supports unsigned integers. This severely limits the scope of the language even for integer computations as sign extension will not be applied. As a result there may be a number of instances where the produced Verilog theoretically matches with the code the system designer wrote, however it does not match with their intentions as they planned to do signed computations. This is extra important if the overflow protections in section \ref{overflow.prot} get implemented, as that would result in addition of a negative number and a positive number to produce the largest unsigned number possible in the given bit width, or from a signed perspective \lstinline|-1|.

It would also be beneficial to introduce fractional types to allow for integral maps and other common stream processing operations which operate on floats. Initially this would be through fixed point, as it is comparatively simple to implement this as an extension of signed integers, however eventually full floating point should be made available. Floating point arithmetic is far more computationally complex than fixed point or integral arithmetic, therefore it would necessitate the multi-cycle optimisation suggested in section \ref{ext.retime} to avoid having an extremely low clock speed.

\section{Used bit analysis}
While variable usage in software programs is limited to whole variables, variables in hardware are usually a single bit, with the exceptions usually coming in the form of shift registers and multipliers. As a result, constant folding and variable usage optimisations can be applied to single bits; this would allow for optimisations based on multiplication by even numbers being even, or further constant folding through slices among others.

\section{Overflow Protection}\label{overflow.prot}
Currently the signal width determination algorithm ignores the concept of overflow, however in the domain of streamed dataflow programming where a value could increase infinitely, it must be considered.

While overflow is sometimes abused by developers to produce `tricks', such as signed addition with unsigned numbers, this is not something which should be enabled by the Flow DSL once support for signed and float types are added. Considering the Flow DSL program in listing \ref{overflow}, when all values are unsigned, it is clear that when \lstinline|z| is greater than \lstinline|0xFFFF|, the output value will be \lstinline|0| even when it is in fact greater. By capping the value of \lstinline|y| to \lstinline|0xFFFF|, or even \lstinline|0x10000|, we honour the input specification by always outputting a high signal if y is greater than \lstinline|0x7FFF|. There are considerations which need to be made for if a capped value is output \hyphen{} is it better to output the maximum representable value or an overflowed value when neither honour the input value? This gets harder when there are other operations which may subtract values ahead of the signal which overflows.

\makeatletter
\AtBeginEnvironment{minted}{\dontdofcolorbox}
\def\dontdofcolorbox{\renewcommand\fcolorbox[4][]{##4}}
\makeatother
\begin{listing}[H]
  \begin{minted}[numbers=left, breaklines]{python}
x = gr.istream(16)
y = x + x
z = gr.ostream(y > gr.const(0x7FFF), 1)
  \end{minted}
  \cprotect\caption{An example where overflow protections are necessary}\label{overflow}
\end{listing}


\section{Improved decoupling}
The final stage of the compilation process and the \lstinline|FlowPrinter| class are currently highly intertwined. While \lstinline|FlowPrinter| does contain a number of concepts which are `pointed' towards Verilog and similar language, such as name generation for unnamed variables and buffers, something which would not be supported by for example LLHD, these concepts would be useful for retargeting to VHDL, Chisel or other tools which do not use SSA in the same way as LLHD or CIRCT.

\section{Adding to xDSL}
The project contains a lot of workarounds for how xDSL works, however to increase the extensibility and potential to reuse, it would be good to add proper support for the operations which needed workarounds to the core xDSL library. These workarounds include ignoring the requirement for programs represented within xDSL to be acyclic \& rewrite patterns to be side-effect free, as discussed in section \ref{dialect.namecheck}.

A better alternative to misusing xDSL and applying workarounds in the future would be to create dedicated subclasses of xDSL optimisation helper classes for applications which require forward references, such as Prolog and the declarative programming model generally, or other types of dataflow programming.

\par\noindent\hrulefill\par

In this section we discussed potential expansions for Flow DSL which, given more time, would have been useful to include in this project. Some alternative methods for the project were also given.

\chapter{Conclusion}
This chapter discusses potential applications of the Flow DSL. It outlines its reusability in hardware DSL development and its potential role as an Intermediate Representation (IR) for hardware compilers focusing on dataflow structure. We end with a reflection on the creation of Flow DSL.

\section{Applications for the Project}
\subsection*{Dialect Reuse}
As xDSL dialects are resusable by design, anyone making a hardware DSL using xDSL, or a dataflow DSL, will be able to reuse any and all of the xDSL dialects created for Flow DSL and introduced in section \ref{chap:compilation}. This could prove useful for anyone making HLS tools \hyphen{} someone working with xDSL in the future could leverage the dialects implemented for Flow DSL as an alternative target, allowing for compilation to hardware of high level software languages.

Similarly, anyone creating a dataflow language for software could simply keep the Flow DSL frontend and stitch it into an existing xDSL dialect which has routes to target software \hyphen{} either some form of bytecode like BEAM, WASM, .NET or JVM, or to machine code which works essentially as an infinite loop, reading inputs and writing outputs in a loop. The optimisations in Flow DSL could be used as a form of software pipelining.

\subsection{Use as an Intermediate Representation}
The Flow DSL can be an effective IR for hardware design, providing a layer where high-level languages can be translated into hardware constructs. Its focus on dataflow structure over timing details enables potential optimizations such as retiming and buffering during the compilation process. However, it should be noted that as of the current state of Flow DSL, it lacks the ability to remove buffers, except in very specific cases. Further research and work may be required to overcome this limitation.

Moreover, the Flow DSL's ability to handle streaming data, by assuming that inputs are populated and outputs are extracted at regular intervals, fits the requirements of an IR in terms of accommodating stream-oriented computations. This makes Flow DSL an ideal candidate for an IR, bridging the gap between high-level programming paradigms and low-level hardware representations.

The dynamic retiming system would also be a useful component of the compilation process of a HLS tool. Unlike static timing, this system provides the flexibility of adjusting the ordering and positions of operations within a pipeline based on the specific requirements of data dependencies, instead of fixed, predetermined timing constraints. This flexibility not only ensures correctness \hyphen{} something HLS tools can have issues with as shown by Herklotz, Pollard, Ramanathan, and Wickerson \cite{formal_verif}, but also allows for the dataflow optimisations provided by Flow DSL to be applied.

\subsection{Retargeting to CIRCT}
The Circuit IR Compilers and Tools (CIRCT) \cite{circt} project, a component of the LLVM ecosystem, provides a novel approach to digital design and circuit compilation. It employs a software compiler infrastructure for hardware design. Given its high-level abstraction capacity, Flow DSL can be retargeted to the CIRCT framework.

Retargeting Flow DSL to CIRCT can potentially provide several benefits. Firstly, it can allow the use of CIRCT's optimization passes on the output RTL code, enhancing the efficiency of the hardware designs aside from dataflow-specific optimisations. Secondly, it can facilitate the integration with other LLVM-based tools, opening routes to more target languages.

\section{Final Remarks}
The journey of creating Flow DSL, a new domain-specific language that simplifies the design of stream processing hardware, has been both challenging and rewarding. We have crafted a product which leverages the flexibility of Python to provide an intuitive and user-friendly way of developing hardware-accelerated stream processing applications.

Flow DSL distinguishes itself through the simplification of complex hardware details, such as the concept of registers and a clock, while still providing a platform for synthesising complex and efficient hardware designs. We have demonstrated how the DSL allows for complex dataflow algorithms, such as generic stencil computers, to be elegantly expressed in an easily comprehensible high-level language, which is subsequently translated into functionally equivalent Verilog code.

This work can be added to the growing body of evidence suggesting that software and hardware languages do not have to be confined to separate worlds, but can interact and learn from each other to produce tools that are both powerful and accessible. Just as we have used the principles of dataflow programming to inform our design of the Flow DSL, we anticipate that future tools will continue to draw from the rich and diverse tapestry of programming paradigms to create innovative and user-friendly solutions.
% Misc.

\printbibliography{}

\end{document}
